\documentclass[a4paper]{scrartcl}
\usepackage[utf8]{inputenc} % use utf8 file encoding for TeX sources
\usepackage[T1]{fontenc}    % avoid garbled Unicode text in pdf
\usepackage[german]{babel}  % german hyphenation, quotes, etc
\usepackage{hyperref}       % detailed hyperlink/pdf configuration
\usepackage{amsfonts}
\usepackage{array} % for "\newcolumntype" macro
\usepackage{pgfplots}
\pgfplotsset{compat=1.8}
\usepackage{pst-plot}
\usepackage{pst-eucl}
\newcolumntype{L}{>{\centering\arraybackslash}m{2.8cm}}

\newcolumntype{M}{>{\centering\arraybackslash}m{5.7cm}}
\usepackage{amsmath}
\graphicspath{ {.} }

\pagestyle{headings}
\pagenumbering{gobble}

\hypersetup{                % ‘texdoc hyperref‘ for options
pdftitle={Analysis 1}
}
\usepackage{graphicx}       % provides comman ds for including figures
\usepackage{csquotes}       % provides \enquote{} macro for "quotes"
\usepackage[nonumberlist]{glossaries}     % provides glossary commands
\usepackage{enumitem}

\title{Analysis Formelsammlung}

\begin{document}
    \maketitle
    \newpage
    \tableofcontents
    \newpage
    
    \section{Komplexe Zahlen}
        \subsection{Definitionen}
            \subsubsection{Definitionen}
                Die Menge der Komplexen Zahlen werden mit dem Symbol  beschrieben.\\
                Sei \(z\) ein Element aus \(\mathbb{C}\) so gilt:
                \begin{equation*}
                    z \, \epsilon \, \mathbb{C} : (z = x + iy \,|\, x,y \,\epsilon \, \mathbb{R})
                \end{equation*}
                Wobei \(x\) der reele Anteil und \(y\) der imaginäre Anteil der komplexen Zahl ist. \\
                \includegraphics{c}
            
            \subsubsection{Formen}    
                \begin{description}
                    \item[Kartesische Form] $z = x + y*i$
                    \item[Trigonometrische Form]   $r(cos\varphi + i*sin\varphi)$
                    \begin{description}
                        \item[Herleitung:]  Es gilt: \(x = r * \cos\varphi, \, y = r * \sin\varphi \\
                        z = x + i*y = r * cos\varphi + i * r* sin\varphi = r(cos\varphi + i*sin\varphi) \), 
                        \\ $r = |z|$ 
                        \item[Berechnung von $\varphi$:] Je nachdem, in welchem Quadranten sich $z$ befindet, ändert sich die Formel leicht um $\varphi$ zu bestimmen.
                        \begin{itemize}
                            \item Quadrant 1: $\varphi = \varphi_{Rechenr}$
                            \item Quadrant 2 : $\varphi = \pi - \varphi_{Rechner}$
                            \item Quadrant 3 : $\varphi = \pi + |\varphi_{Rechner}|$
                            \item Quadrant 4: $\varphi = \varphi_{Rechenr}$
                        \end{itemize}
                    \end{description}
                    \newpage
                    \item[Exponentialform] $r * e^{i\varphi}$ 
                    \begin{description}
                        \item[Herleitung:] Es gilt:\(e^{i\varphi} = cos\varphi + i * sin\varphi \) \\ \(z = r * (e^{i\varphi} = cos\varphi + i * sin\varphi) = r * e^{i\varphi} \), $\varphi$ wird wie oben berechnet.
                    \end{description}
                \end{description}          

            \subsubsection{Grundrechenarten}
                \begin{description}
                    \item[Addieren] $z_1 + z_2 = x_1 + x_2 + (y_1 + y_2)*i$ 
                    \item[Subtrahieren] $z_1 - z_2 = x_1 - x_2 + (y_1 - y_2)*i$ 
                    \item[Multiplizieren] beim multiplizieren wird unterschieden ob mit einem konstanten Faktor oder einer weiteren komplexen Zahl multipliziert wird.
                    \begin{itemize}
                        \item mit einem Faktor: $a * z = a * x + a * y * i $
                        \item mit einer Komplexen Zahl: $z_1 * z_2 = x_1 * x_2 - y_1 * y_2 + (x_1 * y_2 + x_2 * y_1) * i $
                    \end{itemize}  
                    \item[Dividieren]  $\cfrac{z_1}{z_2} = \cfrac{x_1*x_2 + y_1*y_2 + (-x_1*y_2 + x_2*y_1)*i}{x_2^{\;\;2} + y_2^{\;\;2}}$
                    \item[Betrag] $|z| = (x^2 + y^2)^\frac{1}{2} = \sqrt{x^2 + y^2}$ 
                    \item[Komplex Konjugiert] $z = x \pm y*i \rightarrow z^* = x \mp y*i$ 
                \end{description}
                
            \subsubsection{Radiezieren}
            Wenn aus einer Komplexen Zahl z die n-te Wurzel gezogen wird, entstehen somit auch n verschieden Wurzeln.
            die allgemeine Formel dafür lautet 
            \begin{equation*}
                z^{\frac{1}{n}} = |z|^{\frac{1}{n}} * e^{i(\frac{\varphi}{n} + \frac{m}{n})* 2\pi}
            \end{equation*}
            Das $n$ bleibt immer konstant und $m$ startet bei 0 und wird bis $m = n-1$ hochgeählt wodurch man die $n$ Wurzel herbekommt,
            ledigliche $\varphi$ und $|z|$ müssen wie oben beschrieben berechnet werden und ggf. muss $z$ in die Exponentialform gebracht werden, um die Berechnungen zu erleichtern.  
        \newpage
    \section{Funktionen}

    \subsection{Nulstellen}
    $f(x)$ hat Nulstellen wenn gilt $x_0$  $f(x_0) = 0$

    \subsection{Symmetrie}
    \begin{description}
        \item Eine Funktion heißt gerade (spiegelsymmetrisch) wenn gilt: $f(-x) = f(x)$ 
        \item Eine Funktion heißt ungerade(punktsymmetrisch): $f(-x) = -f(x)$
    \end{description}
    
    \subsection{Monotonoie} 
    Die Definition für die Monotonie wird unten aufgelistet. Es gilt außerdem: $x_1 < x_2 $ 
    \begin{itemize}
        \item monoton wachsen:  $f(x_1) \leq f(x_2)$
        \item streng monton wachsend: $f(x_1) < f(x_2)$
        \item monton fallend:  $f(x_1) \geq f(x_2)$
        \item streng monton fallend:  $f(x_1) > f(x_2)$
    \end{itemize}
    
    \subsection{Periodizät}
    Wenn ein $p$ exisitert mit dem gilt $f(x \pm p) = f(x)$ und $x \pm p$ ist im Definitionsbereich, ist $f$ periodisch mit der Periode $p$,\\
    $p$ kann auch $\pm k \, * \,p$ sein, wobei $k \, \epsilon \, \mathbb{N}^*$. Kleinstes positives $p$ nennt man die primitive Periode
    
    \subsection{Umkehrfunktion}
    Funktion f heißt umkehrbahr wenn gilt:     
    \begin{description}
        \item $x_1 \neq x_2 \rightarrow f(x_1) \neq f(x_2)$ oder
        \item wenn f streng monoton ist.
        \end{description}
    Definitions- und Wertebereich sind bei der Umkehrfunktion "vertauscht". \\  
    \\ Um die Umkehrfunktion zu bestimmen, formt man $f(x)$ nach$x$ um und vertauscht danach $x$ und $y$ wodurch man $f^{-1}$ erhält.
    Oft muss der Definitionsbereich dabei eingeschrängt werden, da z.B. ist die Parabel nur für $x \geq 0$ monton ist (bzw. $x \leq 0$). 

    \subsection{Grenzwert einer Funktion}
    Exestiert ein $g$ für das gilt:
    \begin{equation*}
        \lim \limits_{n \to \infty}f(x_n) = g
    \end{equation*} 
    so ist $g$ der Grenzwert von $f$.
    $x_n$ wird durch den limes immer höher, $g$ kann auch gegen unendlich gehen und $g$ muss auch nicht im Wertebereich . Wenn $g$ nicht inm Wertebereich liegt, so ist $g$ auch die Asymptote. \\
    $\lim$ kann auch gegen $-\infty$ oder auch gegen einen Punkt gehen.

    \subsection{Stetigkeit einer Funktion}

    \newpage
    \subsection{Tabelle mit Funktionen und deren Eigenschaften}
        \subsubsection{Polynomfunktion}        
        \begin{description}
            \item[Definition] $f(x) = a_n*x^n...a_1*x + a_0$ wobei $n$ auch der Grad der Funktion ist
            \item[Symmetrie] Eine Polynomfunktion ist gerade, wenn alle Polynome eine gerad Potenzen haben und ist ungerade, wenn alle Polynome ungerade Exponente haben.
        \end{description}

        \subsubsection{Gebrochenrationale Funktion}
        \begin{description}
            \item[Funktion] $f(x) = \cfrac{g(x)}{h(x)}= \cfrac{a_m*x^m...a_1*x + a_0}{b_n*x^n...b_1*x + b_0}$ $g(x)$ unf $h(x)$ sind wiedereum Polynomfunktionen. Außerdem wird unterschieden zwischen:
            \begin{itemize}
                \item $n > m$: echt gebrochenrational und
                \item $n \leq m$ unecht gebrochenrational
            \end{itemize}   
            \item[Nullstellen] $f(x)$ hat Nulstellen wo $g(x)$ Nulstellen hat aber $h(x) \neq 0$ ist.  
            \item[Pol] $f(x)$ hat Pole wo $h(x)$ Nulstellen hat. Wichtig ist, dass zuerst umgerechnet werden muss, da sich teilweise Terme aus dem Bruch rauskürzen könnten.
            Die Anzahl der Nullstellen in $h(x)$ wird mit $k$ bezeichnet und man spricht von eimem Pol $k$-ter Ordnung. Ist $k$
            gerade, gibt es keinen Vorzeichenwechsel am Pol, ist k ungerade gibt es einen Vorzeichenwechsel.
             \item[Asymptote] Um die Asymptote zu finden muss zuerst zwischen einer echten und unechten gebrochenrationalen Funktion unterschieden werden.
                \begin{itemize}
                    \item Echt gebrochenrational: hat eine Asymptote bei $y$ = 0.
                    \item Unecht gebrochenrational: $f(x)$ wird durch Polynomdivison in seine Linearkombination aufgeteilt. $f(x)$ kann nun in eine Polynomfunktion $p(x)$ und
                    eine echt gebrochenrationale Funktion $r(x)$ augeteilt werden  
                    \begin{equation*}
                        f(x) = p(x) + r(x)
                    \end{equation*}
                    $r(x)$ strebt gegen 0 weshalb die Asymptote von $p(x)$ gleich der Asymptote von $f(x)$ ist.
                \end{itemize}
             \item[Symmetrie] je nachdem welche Symmetire die Polynomfunktione haben ist auch die gebrochenrationale Funktoin eine andere Symmetrie. 
                \begin{itemize}
                    \item Haben $g(x)$ und $h(x)$ die gleiche Symmetrie ist $f(x)$ auch gerade.
                    \item Haben $g(x)$ und $h(x)$ verschiedene Symmetrien so ist $f(x)$ ungerade.
                    \item Haben entweder $g(x)$ oder $h(x)$ (oder beide) keine Symmetire so hat $f(x)$ auch keine Symmetrie.
                \end{itemize}  
        \end{description}
        
        \subsubsection{Potenzfunktion}
           \begin{description}
               \item[Funktion] $f(x) = x^n$ 
               \item[Symmetrie] ist n gerade dann ist auch $f(x)$ gerade
               \begin{itemize}
                   \item Ist n gerade dann so ist auch $f(x)$ gerade
                   \item Ist n ungerade dann so ist auch $f(x)$ ungerade
               \end{itemize} 
           \end{description}       
    \newpage

    \section{Differntialrechnung}
        \subsection{Ableitungregeln Funktion}
            \setlength{\arrayrulewidth}{0.5mm}
            \setlength{\tabcolsep}{16pt}
            \renewcommand{\arraystretch}{1.5}
            $\begin{array}{|l|c|c|}    
                    \hline
                    \textbf{Art der Funktion} & \textbf{Funktion} & \textbf{Ableitung}  \\
                    \hline
                    \hline
                    \text{Konstante Funktion}
                        & f(x) = c & f'(x) = 0 \\
                        \hline
                    \text{Gerade}              
                        & f(x) = x & f'(x) = 1 \\
                        \hline
                    \text{Potenzfunktion}      
                        & f(x) = x^n & f'(x) = n*x^{n-1} \\
                        \hline
                    \text{Exponentialfunktion}
                        & f(x) = a^x & f'(x) = \cfrac{1}{ln(a) * a^x} \\
                        & f(x) = a^x = e^{xln(a)} & f'(x) = e^{ln(a)*x} * ln(a) \\
                        & f(x) = x^x = e^{x*ln(x)} & f'(x) = e^{x*ln(x)} * (ln(x)+1) \\
                        \hline
                    \text{Logarithmusfunktion} 
                        & f(x) = ln(x) & f'(x) = \cfrac{1}{x} \\
                        & f(x) = log_a(x) & f'(x) = \cfrac{1}{ln(a)*x} \\
                        \hline
                    \text{Trigonometrische Funktionen} 
                        & f(x) = sin(x) & f'(x) = cos(x) \\
                        & f(x) = cos(x) & f'(x) = -sin(x) \\
                        & f(x) = tan(x) & f'(x) = \cfrac{1}{cos^2(x)} \\
                        & f(x) = cotan(x) & f'(x) = -\cfrac{1}{sin^2(x)} \\
                        \hline
                
                    \text{Arkusfunktionen} 
                        & f(x) = arsin(x) & f'(x) = \cfrac{1}{\sqrt{1-x^2}} \\
                        & f(x) = arcos(x) & f'(x) = -\cfrac{1}{\sqrt{1-x^2}} \\
                        & f(x) = artan(x) & f'(x) = \cfrac{1}{1+x^2} \\
                        & f(x) = arcocotan(x) & f'(x) = -\cfrac{1}{1+x^2} \\
                        \hline
                \text{Hyperbolfunktion}    
                        & f(x) = sinh(x) & f'(x) = cosh(x) \\
                        & f(x) = cosh(x) & f'(x) = -sinh(x) \\
                        & f(x) = tanh(x) & f'(x) = \cfrac{1}{cosh^2(x)} \\
                        & f(x) = cotan(x) & f'(x) = -\cfrac{1}{sinh^2(x)} \\
                        \hline
                \text{Areafunktion} 
                        & f(x) = arsinh(x) & f'(x) = \cfrac{1}{\sqrt{1+x^2}}) \\
                        & f(x) = arcosh(x) & f'(x) = \cfrac{1}{\sqrt{1-x^2}} \\
                        & f(x) = artanh(x) & f'(x) = \cfrac{1}{1-x^2} \\
                        & f(x) = arcotan(x) & f'(x) = \cfrac{1}{1-x^2} \\
                        \hline
            \end{array}$
            \newpage  
        \subsection{Ableitungsregeln Arithmetik}
                \begin{tabular}{|c|c|M|}
                    \hline
                    \textbf{Regeln}  & \textbf{Funktion} & \textbf{Ableitung}  \\
                    \hline
                    \hline
                    \text{Summenregel }
                        & f(x) = u(x) + v(x) & f(x)' = u(x)' + v(x)' \\
                    \hline
                    \text{Faktorregel}              
                        & f(x) = c(x) * u(x) & f(x)' = c(x) * u(x)' \\
                    \hline
                    \text{Produktregel}      
                        & f(x) = u(x) * v(x) & f(x)' = u(x)' * v(x) + u(x) * v(x)' \\
                    \hline
                    \text{Quotientenregel}
                        &  $f(x) = \cfrac{u(x)}{v(x)}$ & $f(x) = \cfrac{ u(x)' * v(x) - v(x) * u'(x)}{v(x)^2}$   \\
                    \hline
                    \text{Kettenregel} 
                        & f(x) = u(v(x)) & f = u'(v(x)) * v(x)' \\
                    \hline
                \end{tabular}
                
                \section{Kurvendiskussion}
                \begin{description}
                    \item[Extrema] \(f'(x) = 0 \wedge f''(x) \neq 0\)
                    \begin{description}
                        \item[Maximum] \(f''(x) < 0\)
                        \item[Minimum]  \(f''(x) > 0\)
                    \end{description}
                    \item[Wendepunkt] \( f'(x) \neq 0 \wedge f''(x) = 0, \)
                    \item[Sattelpunkt] \(f'(x) = f''(x) = 0\) 
                \end{description}
                \section{Integralrechnung}
\end{document}