\documentclass[a4paper,10pt]{scrartcl}
\usepackage[german]{babel}  % german hyphenation, quotes, etc
\usepackage{hyperref}       % detailed hyperlink/pdf configuration
\usepackage{amsfonts}
\usepackage{amsmath}
\usepackage{array} % for "\newcolumntype" macro
\usepackage{pgfplots}
\usepackage{afterpage}
\usepackage{multirow}
\renewcommand{\arraystretch}{1.3}

\title{Analysis}
\pagestyle{headings}
\hypersetup{                % ‘texdoc hyperref‘ for options
pdftitle={Analysis 1}
}

\author{Arif Hasanic}

\begin{document}
    \maketitle
    \afterpage{\null\newpage}
    \newpage
    \tableofcontents
    \newpage

    \section{Differentialgelichungen allgemein}
        Eine Differentialgleichung (DGL) ist eine Funktion in der Ableitung von genau dieser Funktion auftreten können und hat die Form: 
        \begin{equation}
            y' + a(x) \cdot y = b(x) 
        \end{equation}
        Ist \(b(x) = 0\) nennt man die DGL eine homgene DGL, ansonsten hat man eine inhomogene DGL. Die Ordnung einer DGL ist glecih der hächsten Aleitung,
        welche in der DGL zu finden ist. \\
        Löst man eine DGL nach der höchsten Ableitung auf\footnote{Falls überhaupt möglich} hat man die DGL in die implizite Form gebracht; ansonsten hat
        man die implizite Form. \\
        Eine DGL kann man entweder allgemein lösen oder man findet eine spezielle/partikuläre Lösung. bei der allgemeinen Lösung bleiben idR. \(n\) Integrationskonstanten
        stehen, wenn \(n\) gleich der Ordnung der DGL ist. Möchte man die partikuläre Lösung, rechnet man zuerst die allgemein Lösung aus. Nun müssen bestimmte Werte 
        vorgeben werden um die Werte der Integrationskonstanten errechnen zu können. Diese Werte heißen auch Anfangswerte und um eine spezielle Lösung zu finden werden
        auch \(n\) Werte gebraucht. 

    \section{Differentialgleichungen 1. Ordnung}
        \subsection{Trennung der Variabeln}
        Ein realtives einfaches Verfahren zum lösen von DGLs 1. Ordnung nennt sich "Trennung der Variablen". Damit man das Verfahren anwenden kann muss die DGL aber
        seperabel sein, also alle \(x\)-Werte und alle \(y\)-Werte müssen jeweils auf einer Seite stehen können. \\
        Da man ein DGL erster Ordnung hat ist die höchste vorkommende Ableitung \(y'\), was man aber auch als \(\frac{dy}{dx}\) schreiben kann. Wenn man jetzt sowohl alle 
        \(x\)-Werte als auch alle \(y-Werte\) auf einer Seite stehen hat, kann man beide nach \(x\) bzw. \(y\) integrieren. Nach der Integration kann alles nach \(y\) umstellen
        und hat die allgemeine Lösung der DGL gefunden. Beispiel:
        \begin{equation}
            \begin{aligned}  
            & y' = \frac{x}{y} \Leftrightarrow y' = x \cdot \frac{1}{y} \Leftrightarrow \frac{dy}{dx} = x \cdot \frac{1}{y}  \Leftrightarrow \\
            & y \; dy = x \; dx  \Leftrightarrow \int y \; dy = \int x \; dx \Leftrightarrow \frac{1}{2} y^2 = \frac{1}{2} x^2 + C \\
            & y = \sqrt{x^2 + 2C}
            \end{aligned}  
        \end{equation}
        
        \subsection{Substitutionsmehtode}
        Bei der Lösung durch Substitution muss man zuerst wider die Gleichung nach \(y'\) auflösen. Nun schauht man von welchen Typ die DGL ist.
        Die zwei Typen sind \(y' = f(ax + bx + c)\) und \(y' = \frac{y}{x}\).
        \subsubsection{Typ: linear}
        Sei \(u\) die Variabel die zur Substitution genommen wird. Zuerst wird \(u\) der Gleichung gleichgesetzt: \(u = ax + by +c\). Wenn man \(u\) nun
        differenziert erhält man \(u' = a + by'\). \(u\) ist von x abhängig, da \(u\) nur von den variabeln \(x\) und \(y\) abhängig ist und \(y\) wiederum 
        nur von \(x\) abhängig ist. Daraus folgt, dass \(u' = \cfrac{du}{dx}\) gilt. 
        \begin{equation}
            u' = \frac{du}{dx} = a + b \cdot y'
        \end{equation}
        \(y'\) kann man wiederum durch \(f(u)\) ersezten, wodurch man eine DGL erhält, welche nur noch von \(u\) abhängig ist: 
        \begin{equation}
            u' = \frac{du}{dx} = a + b \cdot f(u)
        \end{equation}
        Diese DGL kann man dann durch Trennung der Variabeln lösen und rücksubstituiert das \(u\) mit den ursprünglichen Werten.
        \subsubsection{Typ: quotient}
        Es gilt dasselbe Prinzip wie bei der linearen Funktion. Man substitutiert nun \(\frac{y}{x} = u\). Dementsprechend gilt auch \(y = x \cdot u\). \\
        Wird dies nun differenziert erhält man \(y' = u + x \cdot u'\). Dabei gilt wiederum\(y' = f(\frac{y}{x}) = f(u)\). Wird dies entsprechend eingesetzt gilt:
        \begin{equation}
            u' = \frac{du}{dx} = \frac{f(u) - u}{x}     
        \end{equation} 

        \subsection{lineare DGL mit Störfunktion}
        Eine lineare Differentialgleichung hat die Form: 
        \begin{equation}
            y' + f(x) \cdot y = g(x)
        \end{equation}
        Hier wird \(g(x)\) acuh als Störfunktion bezeichnet. Ist \(g(x) = 0\) ist die DGL homgen, ansonten ist sie inhomogen. Eine homogene DGL lässt sich 
        durch Trennung der Variabeln lösen. Dazu gibt es auch eine allgemein Lösungsform:
        \begin{equation}    
            \begin{aligned}
                & y' + f(x) \cdot y = 0 \Rightarrow \frac{dy}{dx} = -f(x) \cdot y \Rightarrow \frac{dy}{y} = -f(x) \; dx\\
                & \Rightarrow \int \frac{dy}{y} = - \int f(x) \; dx \Rightarrow ln|y| = - \int f(x) \; dx + ln|C| \\
                & \Rightarrow ln |y| - ln|C| = - \int f(x) \; dx \Rightarrow ln|\frac{y}{C} = \int f(x) \; \Rightarrow y - C = \int f(x) \; dx \\
                & \Rightarrow y = C \cdot e^{- \int f(x) \; dx}    
            \end{aligned}
        \end{equation}  
        Tauchen in der DGL konstante Vorfaktoren auf muss die Lösungsformel noch leicht verändert werden:
        \begin{equation}
            y' + ay = 0 \Rightarrow y_h = C \cdot e^{-ax}
        \end{equation}
        \subsubsection{Variation der Konstanten}

        \subsubsection{Aufsuchen der partikulären Lösung}
        Inhomogene Differentialgleichungen (auch höherer Ordnung) können auch durchs Aufuschen der partikulären Lösung gelöst. Die Lösung einer DGL ist dann die
        Summer zwischen der homogenen Lösung \(y_0\)/\(y_h\)  und der partikulären Lösung \(y_p\), also:
        \begin{equation}
            y = y_h + y_p
        \end{equation}
        Zuerst wird die homogene Lösung berechnent. Nun muss noch der richtige Ansatz gewählt werden um die partikuläre Lösung zu finden. \\
        \begin{tabular}{|p{6cm}|p{6cm}|}
            \hline
            Störfunktion \(g(x)\) & Lösungsansatz \(y_p(x)\)  \\
            \hline
            \hline
            1. Konstante Funktion & \(y_p = C_0\) \\
            \hline
            2. Lineare Funktion & \(y_p = C_1 x + C_0\) \\
            \hline
            3. Quadratische Funktion & \(y_p = C_2x^2 + C_1x + C_0\) \\
            \hline
            4. Polynom Funktion mit Grad n & \(y_p = C_nx^n + ... +  C_1x + C_0 \) \\
            \hline
            5. \(g(x) = C_1 \cdot \sin(\omega x)\) & 
            \multirow{3}{6cm}{\(y_p = C_1 \cdot \sin (\omega x) + C_2 \cdot \cos (\omega x) \) \\ oder \\ \(y_p = C \cdot \sin (\omega x + \varphi)\)}  \\
            6. \(g(x) = C_2 \cdot \cos(\omega x)\) \\ 
            7. \(g(x) = C_1 \cdot \sin(\omega x) + C_2 \cdot \cos(\omega x) \) \\
            \hline
            \multirow{2}{6cm}{8. \(g(x) = A \cdot e^{bx} \) } & \multirow{2}{6cm}{\(y_p = C \cdot e^{bx} \text{ für } b \neq -a \) \\ \(y_p = Cx \cdot e^{bx} \text{ für } b = -a\) } \\ 
            & \\
            \hline
        \end{tabular}\\
        Um nun die partikulöre Lösung zu finden nimmt den gefundenen Ansatz her und leitet diesen ab. \(y_p'\) und \(y_p\) werden nun in die ursprüngliche, inhomogene
        Gleichung eingesetzt. Nun muss man nur noch umformen und einen Koeffizientenvergleich vornehmen um die unbestimmeten Konstanten\footnote{Damit sind die \(C_n\) aus der Tabelle gemeint} zu finden. \\
        Beim Koeffizientenvergleich schaut man auf beiden seiten, was als Vorfaktoren bei den \(x\)-Werten steht. Beispiel: 
        \begin{equation*}
            2C_1x^2 + (2C_1 + 2C_2)x + (C_2+ 3C_3) = 2x^2 + 0 \cdot x - 4 
        \end{equation*}
        Auf beiden Seiten steht ein \(x^2\). Hier sihet man auch dass \(C_1 = 1\) sein muss damit die Koeffizienten auf beiden Seiten übereinstimmen. Den "Vergleich" führt man 
        nun mit allen \(C_n\) durchgeführt. Zum Schluss werden die \(C_n\)-Werte in den zuvor gewählten Ansatz eingefügt und man erhält die partikuläre Lösung.  
\end{document}