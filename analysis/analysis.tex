\documentclass[a4paper,10pt]{scrartcl}
\usepackage[german]{babel}  % german hyphenation, quotes, etc
\usepackage{hyperref}       % detailed hyperlink/pdf configuration
\usepackage{amsfonts}
\usepackage{amsmath}
\usepackage{array} % for "\newcolumntype" macro
\usepackage{pgfplots}
\usepackage{afterpage}
\usepackage{multirow}
\renewcommand{\arraystretch}{1.3}
\pdfinfoomitdate=1
\pdftrailerid{}
\title{Analysis}
\date{WS 20/21}
\pagestyle{headings}
\hypersetup{                % ‘texdoc hyperref‘ for options
pdftitle={Analysis 2}
}

\author{Arif Hasanic}

\begin{document}
    \maketitle
    \afterpage{\null\newpage}
    \newpage
    \tableofcontents
    \newpage

    \section{Differentialgleichungen allgemein}
        Eine Differentialgleichung (DGL) ist eine Funktion in der Ableitung von genau dieser Funktion auftreten können und hat die Form: 
        \begin{equation}
            y' + a(x) \cdot y = b(x) 
        \end{equation}
        Ist \(b(x) = 0\) nennt man die DGL eine homogene DGL, ansonsten hat man eine inhomogene DGL. Die Ordnung einer DGL ist gleich der höchsten Ableitung,
        welche in der DGL zu finden ist. \\
        Löst man eine DGL nach der höchsten Ableitung auf\footnote{Falls überhaupt möglich} hat man die DGL in die explizite Form gebracht; ansonsten hat
        man die implizite Form. \\
        Eine DGL kann man entweder allgemein lösen oder man findet eine spezielle/partikuläre Lösung. Bei der allgemeinen Lösung bleiben idR. \(n\) Integrationskonstanten
        stehen, wenn \(n\) gleich der Ordnung der DGL ist. Möchte man die partikuläre Lösung, rechnet man zuerst die allgemeine Lösung aus. Nun müssen bestimmte Werte 
        vorgeben werden um die Werte der Integrationskonstanten errechnen zu können. Diese Werte heißen auch Anfangswerte und um eine spezielle Lösung zu finden werden
        auch \(n\) Werte gebraucht. 

    \section{Differentialgleichungen 1. Ordnung}
        \subsection{Trennung der Variablen}
        Ein relatives einfaches Verfahren zum Lösen von DGLs 1. Ordnung nennt sich "Trennung der Variablen". Damit man das Verfahren anwenden kann muss die DGL aber
        separabel sein, also alle \(x\)-Werte und alle \(y\)-Werte müssen jeweils auf einer Seite stehen können. \\
        Da man ein DGL erster Ordnung hat, ist die höchste vorkommende Ableitung \(y'\), was man aber auch als \(\frac{dy}{dx}\) schreiben kann. Wenn man jetzt sowohl alle 
        \(x\)-Werte als auch alle \(y\)-Werte auf einer Seite stehen hat, kann man beide nach \(x\) bzw. \(y\) integrieren. Nach der Integration kann alles nach \(y\) umstellen
        und hat die allgemeine Lösung der DGL gefunden. Beispiel:
        \begin{equation}
            \begin{aligned}  
            & y' = \frac{x}{y} \Leftrightarrow y' = x \cdot \frac{1}{y} \Leftrightarrow \frac{dy}{dx} = x \cdot \frac{1}{y}  \Leftrightarrow \\
            & y \; dy = x \; dx  \Leftrightarrow \int y \; dy = \int x \; dx \Leftrightarrow \frac{1}{2} y^2 = \frac{1}{2} x^2 + C \\
            & y = \sqrt{x^2 + 2C}
            \end{aligned}  
        \end{equation}
        
        \subsection{Substitutionsmethode}
        Bei der Lösung durch Substitution muss man zuerst wieder die Gleichung nach \(y'\) auflösen. Nun schaut man, von welchen Typ die DGL ist.
        Die zwei Typen sind \(y' = f(ax + by + c)\) und \(y' = \frac{y}{x}\).
        \subsubsection{Typ: linear}
        Sei \(u\) die Variabel die zur Substitution genommen wird. Zuerst wird \(u\) der Gleichung gleichgesetzt: \(u = ax + by +c\). Wenn man \(u\) nun
        differenziert erhält man \(u' = a + by'\). \(u\) ist von x abhängig, da \(u\) nur von den Variablen \(x\) und \(y\) abhängig ist und \(y\) wiederum 
        nur von \(x\) abhängig ist. Daraus folgt, dass \(u' = \cfrac{du}{dx}\) gilt. 
        \begin{equation}
            u' = \frac{du}{dx} = a + b \cdot y'
        \end{equation}
        \(y'\) kann man wiederum durch \(f(u)\) ersetzen, wodurch man eine DGL erhält, welche nur noch von \(u\) abhängig ist: 
        \begin{equation}
            u' = \frac{du}{dx} = a + b \cdot f(u)
        \end{equation}
        Diese DGL kann man dann durch Trennung der Variablen lösen und rücksubstituiert das \(u\) mit den ursprünglichen Werten.
        \subsubsection{Typ: quotient}
        Es gilt dasselbe Prinzip wie bei der linearen Funktion. Man substituiert nun \(\frac{y}{x} = u\). Dementsprechend gilt auch \(y = x \cdot u\). \\
        Wird dies nun differenziert erhält man \(y' = u + x \cdot u'\). Dabei gilt wiederum \(y' = f(\frac{y}{x}) = f(u)\). Wird dies entsprechend eingesetzt gilt:
        \begin{equation}
            u' = \frac{du}{dx} = \frac{f(u) - u}{x}     
        \end{equation} 

        \subsection{lineare DGL mit Störfunktion}
        Eine lineare Differentialgleichung hat die Form: 
        \begin{equation}
            y' + f(x) \cdot y = g(x)
        \end{equation}
        Hier wird \(g(x)\) auch als Störfunktion bezeichnet. Ist \(g(x) = 0\) ist die DGL homogen, ansonsten ist sie inhomogen. Eine homogene DGL lässt sich 
        durch Trennung der Variablen lösen. Dazu gibt es auch eine allgemein Lösungsform:
        %nochmals nachschauen TODO
        \begin{equation}    
            \begin{aligned}
                & y' + f(x) \cdot y = 0 \Rightarrow \frac{dy}{dx} = -f(x) \cdot y \Rightarrow \frac{dy}{y} = -f(x) \; dx\\
                & \Rightarrow \int \frac{dy}{y} = - \int f(x) \; dx \Rightarrow ln|y| = - \int f(x) \; dx + ln|C| \\
                & \Rightarrow ln |y| - ln|C| = - \int f(x) \; dx \Rightarrow ln|\frac{y}{C} = \int f(x) \; \Rightarrow y - C = \int f(x) \; dx \\
                & \Rightarrow y = C \cdot e^{- \int f(x) \; dx}    
            \end{aligned}
        \end{equation}  
        Tauchen in der DGL konstante Vorfaktoren auf muss die Lösungsformel noch leicht verändert werden:
        \begin{equation}
            y' + ay = 0 \Rightarrow y_h = C \cdot e^{-ax}
        \end{equation}
        \subsubsection{Variation der Konstanten}

        \subsubsection{Aufsuchen der partikulären Lösung}
        Inhomogene Differentialgleichungen (auch höherer Ordnung) können auch durchs Aufsuchen der partikulären Lösung gelöst werden. Die Lösung einer DGL ist dann die
        Summe zwischen der homogenen Lösung \(y_0\)/\(y_h\)  und der partikulären Lösung \(y_p\), also:
        \begin{equation}
            y = y_h + y_p
        \end{equation}
        Zuerst wird die homogene Lösung berechnet. Nun muss noch der richtige Ansatz gewählt werden um die partikuläre Lösung zu finden. \\
        \begin{tabular}{|p{6cm}|p{6cm}|}
            \hline
            Störfunktion \(g(x)\) & Lösungsansatz \(y_p(x)\)  \\
            \hline
            \hline
            1. Konstante Funktion & \(y_p = C_0\) \\
            \hline
            2. Lineare Funktion & \(y_p = C_1 x + C_0\) \\
            \hline
            3. Quadratische Funktion & \(y_p = C_2x^2 + C_1x + C_0\) \\
            \hline
            4. Polynom Funktion mit Grad n & \(y_p = C_nx^n + ... +  C_1x + C_0 \) \\
            \hline
            5. \(g(x) = C_1 \cdot \sin(\omega x)\) & 
            \multirow{3}{6cm}{\(y_p = C_1 \cdot \sin (\omega x) + C_2 \cdot \cos (\omega x) \) \\ oder \\ \(y_p = C \cdot \sin (\omega x + \varphi)\)}  \\
            6. \(g(x) = C_2 \cdot \cos(\omega x)\) \\ 
            7. \(g(x) = C_1 \cdot \sin(\omega x) + C_2 \cdot \cos(\omega x) \) \\
            \hline
            \multirow{2}{6cm}{8. \(g(x) = A \cdot e^{bx} \) } & \multirow{2}{6cm}{\(y_p = C \cdot e^{bx} \text{ für } b \neq -a \) \\ \(y_p = Cx \cdot e^{bx} \text{ für } b = -a\) } \\ 
            & \\
            
            \hline
        \end{tabular}\\
        Um nun die partikuläre Lösung zu finden nimmt den gefundenen Ansatz her und leitet diesen ab. \(y_p'\) und \(y_p\) werden nun in die ursprüngliche, inhomogene
        Gleichung eingesetzt. Nun muss man nur noch umformen und einen Koeffizientenvergleich vornehmen um die unbestimmten Konstanten\footnote{Damit sind die \(C_n\) aus der Tabelle gemeint} zu finden. \\
        Beim Koeffizientenvergleich schaut man auf beiden Seiten, was als Vorfaktoren bei den \(x\)-Werten steht. Beispiel: 
        \begin{equation*}
            2C_1x^2 + (2C_1 + 2C_2)x + (C_2+ 3C_3) = 2x^2 + 0 \cdot x - 4 
        \end{equation*}
        Auf beiden Seiten steht ein \(x^2\). Hier sieht man auch, dass \(C_1 = 1\) sein muss damit die Koeffizienten auf beiden Seiten übereinstimmen. Den "Vergleich" führt man 
        nun mit allen \(C_n\) durchgeführt. Zum Schluss werden die \(C_n\)-Werte in den zuvor gewählten Ansatz eingefügt und man erhält die partikuläre Lösung.  
    \subsection{Bernoulli DGL}
        Bernoulli Differentialgleichungen haben die Form 
        $$y' = f(x)\cdot y + g(x) \cdot y^\alpha$$
        Dabei substituiert man $z = y^{1-\alpha}$ , was zu 
        $$z' = (1-\alpha)\cdot f(x)\cdot z + (1-\alpha)\cdot g(x)$$
        führt und als inhomogene DGL gelöst werden kann.
    \section{Lineare Differentialgleichungen 2. Ordnung}
        Eine DGL 2. Ordnung löst man weitesgehend wie eine eine DGL 1. Ordnung, indem man wieder die homogene und die partikuläre Lösung findet und aus diesen  die Summe bildet:
        \begin{equation}
            \begin{aligned}
                & y'' + ay' + by = g(x) \\
                & \Rightarrow y(x) = y_h(x) + y_p(x) 
            \end{aligned}
        \end{equation}   
        \subsection*{Homogene Lösung}
        Bei der homogenen Lösung wählt man vorerst den \(y = e^{\lambda x}\). Dies wird dann in die DGL eingesetzt und Entsprechend oft abgeleitet (2 Mal also).
        \begin{equation}
            \begin{aligned}
                & y'' = a y' + by = 0 \\
                & \Rightarrow e^{\lambda x} \;^{\prime \prime}  + a e^{\lambda x} \;^{\prime} + e^{\lambda x} = 0 \\
                & \Rightarrow \lambda^2 e^{\lambda x}  + a\lambda e^{\lambda x} + be^{\lambda x} = 0 & |\text{Division durch \(e^{\lambda x}\) } \\
                & \Rightarrow \lambda^2 + a \lambda + b = 0
            \end{aligned}
        \end{equation} 
        Man erhält somit einer Formel die man durch die pq-Formel oder der Mitternachtsformel lösen kann. Je nachdem welche Werte \(\lambda_1\) und \(\lambda_2\) haben,
        muss ein andere Ansatz gewählt werden. \\
        \begin{tabular}{|p{3cm}|p{6cm}|}
            \hline 
            Fall & Allgemeine Lösung \\
            \hline
            1. \(\lambda_1 \neq \lambda_2\) & \(y = C_1 \cdot e^{\lambda_1 x} + C_2 \cdot e^{\lambda_2 x}\) \\
            \hline
            2. \(\lambda_1 = \lambda_2 = c\) & \(y = (C_1 + C_2x) e^{cx}\) \\
            \hline
            3. \(\lambda_{1/2} = a + bi \) & \(y = e^{ax}( C_1 \cdot \sin (bx) + C_2 \cdot \cos (bx) )\) \\
            \hline
        \end{tabular} \\
        Die Werte die man für \(\lambda_{1/2}\) berechnet hat werden nun in den richtigen Ansatz eingefügt, wodurch man die homogene Lösung bekommt.  
        
        \subsection*{Inhomogene Lösung}
        Die inhomogene Lösung berechnet sich wie die inhomogene Lösung bei einer DGL 1. Ordnung, nur muss man noch u.U. die Werte von \(\lambda_{1/2}\) berücksichtigen.
        %TODO: Tabelle mit Störfunktionen 
    
        \section{Differentialgleichungen n-ter Ordnung}
        \subsection*{Homogene Lösung}
        
        Für die Lösung wird wieder das Prinzip aus der DGL 2. Ordnung genommen. Bei der homogenen Lösung wird nun eine Linearkombination aus \(C_n \cdot \lambda^n \) und 
        zwar aus allen \(n\) gebildet. Sind die Nullstellen gefunden, werden diese dann in den Ansatz eingefügt. Mehrfache Nullstellen werden zusammengefasst: \\
        Hat man eine \(r\)-Fache Nullstelle multipliziert man den \(e^{\lambda x}\) Teil mit einem Polynom (\(r-1\))-sten Grades. \\
        Als beispiel: Man hat eine DGL 5. Grades und eine 4-Fache Nullstelle. Somit ergibt sich der Ansatz 
        \begin{equation}
            y = C_1 \cdot e^{\lambda_1} + (C_2 + C_3x + C_4x^2 + c_5x^3) \cdot e^{x \cdot \lambda_{2/3/4/5}}
        \end{equation}   

        \subsection*{Inhomogene Lösung}
        %TODO: Tabelle

        \section{Systeme linearer Differentialgleichungen}
        Systeme von Differentialgleichungen sind DGLen mit verschiedenen \(y\) (Es extstieren  \(y_1, y_2, y_3\) etc.), die aber miteinander verkoppelt sind.
        \begin{equation*}
            \begin{aligned}
                & y_1' = a_{11}y_1 + a_{12}y_2 + g_1(x) \\
                & y_2' = a_{21}y_2 + a_{22}y_1 + g_2(x)  
            \end{aligned}
        \end{equation*}
        Für die Lösung von DGL Systemen gibt es 2 Lösungsansätze.

        \subsection*{Homogene Lösung durch Matrix}
            Mit dem Matrixverfahren kann die homogene Lösung gefunden werden. Beim Matrixverfahren werden aus
            den Koeffizienten eine Matrix gebildet. Die Koeffizientenmatrix wird mit \(\lambda \cdot E\footnote{Einheitsmatrix}\) subtrahiert und daraus wird die Determinante gebildet.
            \(\lambda\) wird so gewählt, dass \(\det (A - \lambda E) = 0\) ergibt. Die \(\lambda\)-Werte werden dann in den allgemeinen Lösungsansatz für homogene DGl n-ter Ordnung eingesetzt.
            \begin{equation*}
                \begin{aligned}
                    & y_1' = a_{11}y_1 + a_{12}y_2 + g_1(x) \\
                    & y_2' = a_{21}y_2 + a_{22}y_1 + g_2(x) \\
                    & \Rightarrow
                    \begin{pmatrix}
                        y_1 \\ y_2
                    \end{pmatrix} = 
                    \begin{pmatrix}
                        a_{11} & a_{12} \\
                        a_{12} & a_{22} 
                    \end{pmatrix} 
                    \begin{pmatrix}
                        y_1 \\ y_2
                    \end{pmatrix} + 
                    \begin{pmatrix}
                        g_1(x) \\ g_2(x)
                    \end{pmatrix} \\
                    & \Rightarrow \det (A - \lambda E))
                \end{aligned} \\
            \end{equation*} 
            Für DGL Systeme 2. Ordnung ergibt sich eine \(2x2\) Matrix und die Determinante lässt sich durch \(\det = \lambda^2 - (a_{11} + a_{22} + a_{11}a_{22} - a_{12}a_{21} = 0)\)
            berechnen. Man erhält dann \(\lambda_{1/2}\) und muss wieder eine Fallunterscheidung wie bei normalen DGLs durchführen.  

        \subsection*{Inhomogene Lösung durch Einsetzverfahren}
            Ein weiteres Verfahren ist das Einsetzverfahren. Man löst beispielsweise die Gleichung von \(y_1\) nach y
            \(y_2\) auf und differenziert \(y_2\) nach \(x\). Die Ergebnisse von \(y_2\) und \(y_2'\) kann man in die 2. Gleichung einsetzen.
            Man erhält eine DGL die nur noch von \(y_1\) abhängig ist und löst diese DGL wie eine inhomogene DGL n-ter Ordnung. Man 
            erhält dann das Ergebnis für \(y_1\), und dieses Ergebnis kann man dann in die andere DGL einsetzen und hat somit das ganze System gelöst.
            
            \begin{equation*}
                \begin{aligned}
                    & y_1' = a_{11}y_1 + a_{12}y_2 + g_1(x) &  (1)\\
                    & y_2' = a_{21}y_2 + a_{22}y_1 + g_2(x) &  (2)\\
                    & y_2 = \frac{1}{a_{12}} (y_1' - a_{11}y_1 - g_1(x)) & \text{(3) ((1) nach  \(y_2\)) } \\
                    & \Rightarrow y_2' = \frac{1}{a_{12}} (y_1'' - a_{11}y_1' - g_1(x)') & (3') \\
                \end{aligned}
            \end{equation*}
            (3) und (3') werden in (2) eingesetzt und (2) wird wie eine normale DGL behandelt. Das Ergebnis von (2) wird dann in (1) eingesetzt.
    
    \section{Vektoralgebra}
        \subsection{Vektordarstellung einer Kurve}
        Kurven in einer Ebene können in der Form: 
        \begin{equation*}
            C: x = x(t); y = y(t)
        \end{equation*}
        Können durch den Kurvenpunkt P (bestimmt durch x(t), y(t)) und durch den Ortsvektor auch as Vektor dargestellt werden:
        \begin{equation*}
            \vec{r} = x(t)\vec{e}_x + y(t)\vec{e}_y = \begin{pmatrix} x(t) \\ y(t) \end{pmatrix}
        \end{equation*}
        Dies gilt für Kurven in der Ebene, für höhere Dimensionen muss der Vektor entsprechend angepasst werden. Vektoren werden Komponentenweiße abgeleitet;
        es gelten die selben Produktregeln\footnote{sowohl Skalarprodukt als auch Vektorprodukt} und Summenregeln. 
        \subsection*{Bogenlänge}
        Die Bogenlänge einer Vektorkurve ist definiert durch das Integral des Betrages des abgeleiteten Vektor:
        \begin{equation*}
            s = {\int_{t1}}^{t2} | \vec{\dot{r}} | dt = {\int_{t1}}^{t2}  \sqrt{{\dot{x}}^2 +{\dot{y}}^2} dt 
        \end{equation*} 
        Um den Tangentenvektor (\(T\)) von \(\vec{r}\) zu berechnen, muss dieser nur normalisiert werden. Der Hauptnormaleinheitsvektor ist der der 
        Vektor \(T\) nur wird dieser nochmals normalisiert.
    \subsection{Partielle Differentiation}
        Bei der partiellen Differentiation liegt eine Funktion vor, die von mehr als einer Variablen abhängig ist. Man wählt nun eine Variable, nach der differenziert
        werden soll, die anderen Variablen werden wie Konstanten behandelt. Ein wichtiger Operator ist der \(\nabla\) (Nabla) Operator. Wird dieser angewendet wird nach jeder 
        Variabel abgeleitet.
        \subsubsection{Gradient}
        Der Gradient (kurz: grad) ist das Produkt aus \(\nabla\) und einem Skalar \(\phi\). Als Ergebnis erhält man einen Vektor, dessen Komponenten aus dem Differentialen 
        von \(\phi\) nach \(x_1, x_2,\cdot,x_n\) besteht. 
        \begin{equation*}
            \text{grad } \phi = \vec{\nabla} \phi = \begin{pmatrix} \frac{\partial}{\partial x} \\ \frac{\partial}{\partial y} \\ \frac{\partial}{\partial z} \end{pmatrix} \phi = 
            \begin{pmatrix} \frac{\partial \phi}{\partial x} \\ \frac{\partial \phi}{\partial y} \\ \frac{\partial \phi}{\partial z} \end{pmatrix} 
        \end{equation*}

        \subsubsection*{Rechenregeln}
        \begin{equation*}
            \begin{aligned}
                & \text{(1) } \text{grad } c = 0 \\
                & \text{(2) } \text{grad } (c \cdot \phi) = c \cdot \text{grad }\phi \\
                & \text{(3) } \text{grad } (\phi + \psi) = \text{grad } \phi + \text{grad } \psi \\
                & \text{(4) } \text{grad } (\phi + c) = \text{grad } \phi \\
                & \text{(5) } \text{grad } (\phi \cdot \psi)= \phi(\text{grad } \psi) + \psi(\text{grad } \phi) \\
                & \text{(5') } \text{grad } (\phi \cdot \psi)= \phi \cdot \text{grad } \psi + \psi \cdot \text{grad } \phi \\
            \end{aligned}
        \end{equation*}
        Desweiteren kann man durch den Gradienten analysieren wie sich die Funktionswerte ändern, wenn man in eine bestimmte Richtung fortschreitet. Dazu nimmt man den
        Vektor \(\vec{a}\) als Richtungsvektor. Dieser wird normiert und dann mit dem Gradientenvektor skalar multipliziert. Das Ergebnis ist dementsprechend ein Skalar. 
        \begin{equation*}
            \frac{\partial \phi}{\partial \vec{a}} = (\text{grad}  \phi) \cdot \vec{e_a} = \frac{1}{|\vec{a}|} (\text{grad } \phi) \cdot \vec{a}
        \end{equation*}
        
        \subsection{Divergenz}
        Bei der Divergenz nimmt man einen Vektor und differenziert dessen Komponenten einzeln (nach \(x_1, x_2,\) etc.), und addiert diese zusammen. 
        \begin{equation*}
            \text{div } \vec{F} = \frac{\partial F_x}{\partial x} + \frac{\partial F_y}{\partial y} +\frac{\partial F_z}{\partial z}
        \end{equation*} 
        \begin{itemize}
            \item div \(\vec{F}\) \(>\) 0: Quelle
            \item div \(\vec{F}\) \(<\) 0: Senke
            \item div \(\vec{F}\) = 0: quellenfrei 
        \end{itemize}

        \subsubsection*{Rechenregeln}
        \begin{equation*}
            \begin{aligned}
                & \text{(1) } \text{div } \vec{a} = 0 \\
                & \text{(2) } \text{div } (c \cdot \vec{A}) = c \cdot \text{div } \vec{A} \\
                & \text{(3) } \text{div } (\vec{A} + \vec{B}) = \text{div } \vec{A} + \text{div } \vec{B} \\
                & \text{(4) } \text{div } (\vec{A} + \vec{a}) = \text{div } \vec{A} \\
                & \text{(5) } \text{div } (\phi \vec{A})= (\text{grad } \phi) \cdot \vec{A} + \phi(\text{div } \vec{A}) \\
                & \text{(5') } \text{div } (\phi \vec{A})= (\text{grad } \phi) \cdot \vec{A} + \phi \cdot \text{div } \vec{A} \\
            \end{aligned}
        \end{equation*}

        \subsection{Rotation}
        Die Rotation ist das Kreuzprodukt zwischen einem Vektor \(\vec{F}\) und \(\nabla\). 
        \begin{equation*}
            \text{rot } \vec{F} = \begin{pmatrix}
                \cfrac{\partial F_z}{\partial y}  - \cfrac{ \partial F_y}{\partial z}
            \end{pmatrix} \vec{e}_x + 
            \begin{pmatrix}
                \cfrac{\partial F_x}{\partial z}  - \cfrac{ \partial F_z}{\partial x}
            \end{pmatrix} \vec{e}_y +
            \begin{pmatrix}
                \cfrac{\partial F_y}{\partial x}  - \cfrac{ \partial F_x}{\partial y}
            \end{pmatrix} \vec{e}_z =            
            \begin{pmatrix}
                \cfrac{\partial F_z}{\partial y}  - \cfrac{ \partial F_y}{\partial z} \\
                \\
                \cfrac{\partial F_x}{\partial z}  - \cfrac{ \partial F_z}{\partial x} \\ \\
                \cfrac{\partial F_y}{\partial x}  - \cfrac{ \partial F_x}{\partial y}
            \end{pmatrix}
        \end{equation*}
        Im ebenen Vektorfeld gilt:
        \begin{equation*}
            \text{rot }\vec{F} = \left(\frac{\partial F_y}{\partial x} - \frac{\partial F_x}{\partial y}\right)\cdot \vec{e}_z
        \end{equation*}
        \subsection*{Rechenregeln}
        \begin{equation*}
            \begin{aligned}
                & \text{(1) rot }  \vec{a} = \vec{0} \\
                & \text{(2) rot }  (c \cdot \vec{A}) = c \cdot \text{rot } \vec{A} \\
                & \text{(3) rot }  (\vec{A} + \vec{B}) = \text{rot } \vec{A} + \text{rot } \vec{B} \\
                & \text{(4) rot }  (\vec{A} + \vec{a}) = \text{rot } \vec{A} \\
                & \text{(5) rot }  (\phi \vec{A})= (\text{grad }  \phi) \times \vec{A} + \phi(\text{rot } \vec{A}) \\
                & \text{(5') rot }  (\phi \vec{A})= (\text{grad }  \phi) \times \vec{A} + \phi \cdot \text{rot } \vec{A} \\
            \end{aligned}
        \end{equation*}

    \section{Koordinatendarstellungen}
        \subsection{Polarkoordinaten}
        Anstatt dass zwei Punkt (\(x \text{ und } y\)) gegeben  sind, kann auch jeder Punkt in einem Koordinatensystem durch einen Radius \(r\) und 
        einem Winkel \(\varphi\) bestimmt werden.
        \subsection*{Polarkoordinaten \(\rightarrow\) Kartesisch Koordinaten}
            \begin{equation*}
                x = r \cdot \cos \varphi\text{, } y = r \cdot \sin \varphi  
            \end{equation*}
        \subsection*{Polarkoordinaten \(\rightarrow\) Kartesische Koordinaten}
            \begin{equation}
                r = \sqrt{x^2 + y^2} \text{, } \sin \varphi = \cfrac{y}{r} \text{, } \cos \varphi = \cfrac{x}{r} \text{, } \tan \varphi = \cfrac{y}{x} 
            \end{equation}
        \subsubsection{Vektorform}
        Ein Vektor in der Form \(\vec{a} a_x\vec{e}_x a_y\vec{e}_y\) (kartesische Form) kann auch durch Polarkoordinaten dargestellt werden.
        Das Resultat ist dann der Vektor der sich durch \(\vec{a} = r\vec{e}_r + a_{\varphi}\vec{e}_{\varphi} \) darstellen lässt. Man erhält den
        Vektor, wenn man eine Transformationsmatrix \(A\) mit ihm multipliziert: 
        \begin{equation*}
            \begin{pmatrix} a_x \\ a_y \end{pmatrix} = \begin{pmatrix} \cos \varphi & \sin \varphi \\ 
                -\sin \varphi & \cos \varphi  \end{pmatrix} 
                \begin{pmatrix} a_x \\ a_y \end{pmatrix} = 
                A \begin{pmatrix} a_x \\ a_y \end{pmatrix}
        \end{equation*}
        \begin{equation*}
            A^{-1} = \begin{pmatrix} \cos \varphi & -\sin \varphi \\ 
                \sin \varphi & \cos \varphi  \end{pmatrix}
        \end{equation*}
        Ohne Matrixmultiplikation: 
        \begin{equation*}
            \begin{aligned}
                & a_r = a_x \cdot \cos \varphi + a_y \cdot \sin \varphi \\
                & a_{\varphi} = -a_x \cdot \sin + a_y \cos \varphi \\
                & \Rightarrow a = a_r + a_{\varphi}
            \end{aligned}
        \end{equation*}
        \(A^{-1}\) multipliziert mit dem Polarvektor ergibt wieder den Vektor in kartesischer Form, bzw. kann man auch die Gleichungen nach \(x\) und 
        \(y\) umstellen.

        \subsubsection{Gradient, Divergenz, Rotation und Laplace Operator in Polarkoordinaten}
        Ein Skalarfeld in Polarform ist gegeben durch \(\phi = \phi(r; \varphi)\) und ein Vektorfeld in Polarform ist gegeben durch \(\vec{F} =
        \vec{F}(r;\varphi) = F_r(r;\varphi)\vec{e}_r + F_{\varphi} (r;\varphi)\vec{e}_{\varphi}\), dann gilt für Gradient, Divergenz, Rotation und 
        den Laplace-Operator folgendes: 
        \begin{itemize}
            \item grad \(\phi(r;\varphi) = \cfrac{\partial \phi}{\partial r}\vec{e}_r + \cfrac{1}{r} \cdot \cfrac{\partial \phi}{\partial \phi}\vec{e}_{\varphi}\) 
            \item div \(\vec{F}(r;\varphi) = \cfrac{\partial}{\partial r}(r \cdot F_r) + \cfrac{1}{r} \cdot \cfrac{\partial F_{\varphi}}{\partial \varphi}  \)
            \item rot \(\vec{F}(r;\varphi) = \cfrac{1}{r} \cdot \cfrac{\partial}{\partial r} (r \cdot F_{\varphi}) - \cfrac{1}{r} \cdot \cfrac{\partial F_r}{\partial \varphi} \)
            \item \(\varDelta \phi (r;\varphi)\footnote{\(\varDelta\) (Laplace) ist \(\nabla\)(Nabla) 2 mal ausgeführt} = \cfrac{\partial^2\phi}{\partial r^2} + \cfrac{1}{r} \cdot \cfrac{\partial \phi}{\partial r} + \cfrac{1}{r^2} \cdot \cfrac{\partial^2 \phi}{\partial \varphi^2} \)
        \end{itemize}
        \subsection{Zylinderkoordinaten}
            Zylinderkoordinaten sind Polarkoordinaten mit einer weiteren Dimension \(z\), wodurch ein Zylinderform entsteht. \(\varrho\) beschreibt den Abstand des eines Punktes zur z-Achse :
            \subsubsection*{Zylinderkoordinaten \(\rightarrow\) Kartesische Koordinaten}
                \begin{equation*}
                    x = \varrho \cdot \cos \varphi \text{, } y = \varrho \cdot \sin \varphi \text{, } z = z
                \end{equation*}
            \subsubsection*{Kartesische Koordinaten \(\rightarrow\) Zylinderkoordinaten}
            \begin{equation}
                \begin{aligned}
                    & \varrho = \sqrt{x^2 + y^2} \\
                    & \sin \varphi = \cfrac{y}{\varrho} \text{, } \cos \varphi = \cfrac{x}{\varrho} \text{, } \tan \varphi = \cfrac{y}{x} \\
                    & z = z
                \end{aligned}
            \end{equation}

            \subsubsection*{Spezielle Werte}
            \begin{itemize}
                \item Linienelement: \(ds\footnote{Das \(d\) ist dann der Abstand zweier Punkte} = \sqrt{(d\varrho)^2 + \varrho^2(d\varphi) + (dz)^2} \)
                \item Flächenelement: \(dA = \varrho \; d\varphi \; dz\)
                \item Volumenelement: \(dV = \varrho \; d\varrho \; d\varphi \; dz\)
            \end{itemize}

            \subsubsection{Vektorform}
                Ein Vektor in kartesischer Form (\( \vec{a}= a_x\vec{e}_x + a_y\vec{e}_y + a_z\vec{e}_z \)) lässt sich durch \( \vec{a}= a_\varrho\vec{e}_\varrho + a_\varphi\vec{e}_\varphi + a_z\vec{e}_z \)
                in Zylinderkoordinatenform darstellen, wobei man \(a_\varrho, a_\varphi, a_z\) wie folgt bestimmen kann: 
                \begin{equation*}
                    \begin{aligned}
                        & a_\varrho = a_x \cdot \cos \varphi + a_y \cdot \sin \varphi \\
                        & a_\varphi = -a_x \cdot \sin \varphi + a_y \cos \varphi \\
                        & a_z = a_z
                    \end{aligned}
                \end{equation*}
                Möchte man das mit einer Matrix lösen gilt folgendes:   
                \begin{equation*}
                    \begin{pmatrix}
                        a_\varrho \\ a_\varphi \\ a_z
                    \end{pmatrix} = 
                    \begin{pmatrix}
                        \cos \varphi & \sin \varphi & 0 \\
                        -\sin \varphi & \cos \varphi & 0 \\
                        0 & 0 & 1
                    \end{pmatrix}
                    \begin{pmatrix}
                        a_x \\ a_y \\ a_z
                    \end{pmatrix} = A 
                    \begin{pmatrix}
                        a_x \\ a_y \\ a_z
                    \end{pmatrix}
                \end{equation*}     
                \begin{equation*}
                    A^{-1} = \begin{pmatrix} \cos \varphi & -\sin \varphi & 0 \\ 
                        \sin \varphi & \cos \varphi & 0 \\
                        0 & 0 & 1  \end{pmatrix}
                \end{equation*}
            \subsubsection{Gradient, Divergenz, Rotation und Laplace Operator in Zylinderkoordinaten}
            Ein Skalarfeld ind Zylinderkoordinatenform ist gegeben durch \(\phi = \phi(\varrho; \varphi; z)\) und ein Vektorfeld in Zylinderkoordinatenform ist gegeben durch \(\vec{F} =
            \vec{F}(\varrho; \varphi; z) = F_\varrho(\varrho; \varphi; z)\vec{e}_\varrho + F_\varphi(\varrho; \varphi; z)\vec{e}_\varphi + F_z(\varrho; \varphi; z)\vec{e}_z \), dann gilt für Gradient, Divergenz, Rotation und 
            den Laplace-Operator folgendes: 
            \begin{itemize}
                \item grad \(\phi(\varrho; \varphi; z) = \cfrac{\partial \phi}{\partial \varrho}\vec{e}_\varrho + \cfrac{1}{\varrho} \cdot \cfrac{\partial \phi}{\partial \varphi} \vec{e}_\varphi + \cfrac{\partial \phi}{\partial z} \vec{e}_z \) 
                \item div \(\vec{F}(\varrho; \varphi; z) = \cfrac{1}{\varrho} \cdot \cfrac{\partial}{\partial \varrho}(\varrho \cdot F_\varrho) + \cfrac{1}{\varrho} \cdot \cfrac{\partial F_\varphi}{\partial \varphi} + \cfrac{\partial F_z}{\partial z} \)
                
                \item rot \(\vec{F}(\varrho; \varphi; z) = 
                \Bigg(  \cfrac{1}{\varrho} \cdot \cfrac{\partial F_z}{\partial \varphi} - \cfrac{\partial F_\varphi}{\partial z} \Bigg)\vec{e}_\varrho + 
                \Bigg(  \cfrac{\partial F_\varrho}{\partial z} - \cfrac{\partial F_z}{\partial \varrho}  \Bigg)\vec{e}_\varphi + 
                \cfrac{1}{\varrho} \Bigg(  \cfrac{\partial}{\partial \varrho}  (\varrho \cdot F_\varphi) - \cfrac{\partial F_\varrho}{\partial \varphi}  \Bigg)\vec{e}_z \)
                
                \item \(\varDelta \phi = \cfrac{1}{\varrho} \cdot \cfrac{\partial}{\partial \varrho} \Bigg( \varrho \cdot \cfrac{\partial \phi}{\partial \varrho} \Bigg) + 
                \cfrac{1}{\varrho^2} \cdot \cfrac{\partial^2 \phi}{\partial \varphi^2} + \cfrac{\partial^2 \phi}{\partial z^2} \)
            \end{itemize}
            
        \subsection{Kugelkoordinaten}
            Kugelkoordinaten sind durch \(r\) (Abstand vom Punkt zum Ursprung), \(\varphi\) (Winkel auf der \(x,y\)-Ebene) und durch \(\vartheta\) (Winkel auf der \(y,z\)-Ebene) definiert.
            \subsubsection*{Zylinderkoordinaten \(\rightarrow\) Kartesische Koordinaten}
                \begin{equation*}
                    x = r \cdot \sin \vartheta \cdot \cos \varphi \text{, } y = r \cdot \sin \vartheta \cdot \sin \varphi \text{, } z = r \cdot \cos \vartheta
                \end{equation*}
            \subsubsection*{Kartesische Koordinaten \(\rightarrow\) Zylinderkoordinaten}
            \begin{equation}
                \begin{aligned}
                    & \varrho = \sqrt{x^2 + y^2 + z^2} \\
                    & \vartheta = \arccos \cfrac{z}{r} = \Bigg( \cfrac{z}{\sqrt{x^2 + y^2 + z^2}} \Bigg) \\
                    & \sin \varphi = \cfrac{y}{\sqrt{x^2 + y^2}} \text{, } \cos \varphi = \cfrac{x}{\sqrt{x^2 + y^2}} \text{, } \tan \varphi = \cfrac{y}{x}  
                \end{aligned}
            \end{equation}
            
            \subsubsection*{Spezielle Werte}
            \begin{itemize}
                \item Linienelement: \(ds = \sqrt{(dr)^2 + r^2(d\vartheta)^2 + r^2 \cdot \sin^2 \vartheta (d\varphi)^2} \)
                \item Flächenelement: \(dA = r^2 \cdot \sin \vartheta \; d\vartheta \; d\varphi \)
                \item Volumenelement: \(dV = dA \; = r^2 \cdot \sin \vartheta \; dr \; d\vartheta \; d\varphi\)
            \end{itemize}
            Um einen Vektor in kartesischer Form (\(\vec{a} = a_x\vec{e}_x + a_y \vec{e}_y + a_z \vec{e}_z \)) umzuwandeln in einen Vektor in Kugelkoordinatenform 
            (\(\vec{a} = a_r\vec{e}_r + a_\vartheta\vec{e}_\vartheta + a_\varphi\vec{e}_\varphi\)) kann folgender Ansatz genommen werden:
            \begin{equation*}
                \begin{aligned}
                    & a_r = a_x \cdot \sin \vartheta \cdot \cos \varphi + a_y \sin \vartheta \cdot \sin \varphi + a_z \cdot \cos \vartheta \\
                    & a_\vartheta = a_x \cos \vartheta \cdot \cos \varphi + a_y \cdot \cos \vartheta \cdot \sin \varphi - a_z \cdot \sin \vartheta \\
                    & a_\varphi = -a_x \cdot \sin \varphi + a_y \cdot \cos \varphi
                \end{aligned}
            \end{equation*}
            bzw. mit Matrixmultiplikation: 
            \begin{equation*}
                \begin{pmatrix}
                    \vec{e}_r \\ \vec{e}_\vartheta \\ \vec{e}_\varphi 
                \end{pmatrix} = 
                \begin{pmatrix}
                    \sin \vartheta \cdot \cos \varphi & \sin \vartheta \cdot \sin \varphi & \cos \vartheta \\
                    \cos \vartheta \cdot \cos \varphi & \cos \vartheta \cdot \sin \varphi & - \sin \vartheta \\
                    -\sin \varphi & \cos \varphi & 0
                \end{pmatrix}
                \begin{pmatrix}
                    \vec{e}_x \\ \vec{e}_y \\ \vec{e}_z
                \end{pmatrix} = A
                \begin{pmatrix}
                    \vec{e}_x \\ \vec{e}_y \\ \vec{e}_z
                \end{pmatrix}
            \end{equation*}
            \begin{equation*}
                \begin{pmatrix}
                    \vec{e}_x \\ \vec{e}_y \\ \vec{e}_z 
                \end{pmatrix} = 
                \begin{pmatrix}
                    \sin \vartheta \cdot \cos \varphi & \cos \vartheta \cdot \cos \varphi & -\sin \varphi \\
                    \sin \vartheta \cdot \sin \varphi & \cos \vartheta \cdot \sin \varphi & \cos \varphi \\
                    \cos \vartheta & -\sin \vartheta & 0
                \end{pmatrix}
                \begin{pmatrix}
                    \vec{e}_r \\ \vec{e}_\vartheta \\ \vec{e}_\varphi 
                \end{pmatrix} = A^{-1}
                \begin{pmatrix}
                    \vec{e}_r \\ \vec{e}_\vartheta \\ \vec{e}_\varphi 
                \end{pmatrix}
            \end{equation*}
            \subsubsection{Gradient, Divergenz, Rotation und Laplace Operator in Kugelkoordinaten}
            Ein Skalarfeld in Kugelkoordinaten ist gegeben durch \(\phi = \phi(r; \vartheta; \varphi)\) und ein Vektorfeld in Kugelkoordinaten ist gegeben durch \(\vec{F} =
            \vec{F}(r; \vartheta; \varphi) = F_r(r; \vartheta; \varphi)\vec{e}_r + F_\vartheta(r; \vartheta; \varphi)\vec{e}_\vartheta + F_\varphi(r; \vartheta; \varphi)\vec{e}_\varphi \), dann gilt für Gradient, Divergenz, Rotation und 
            den Laplace-Operator folgendes: 
            \begin{itemize}
                \item grad \(\phi = \cfrac{\partial \phi}{\partial r}\vec{e}_r + \cfrac{1}{r} \cdot \cfrac{\partial \phi}{\partial \vartheta}\vec{e}_\vartheta + \cfrac{1}{r \cdot \sin \vartheta} \cdot \cfrac{\partial \phi}{\partial \varphi}\vec{e}_\varphi\)  
                \item div \(\vec{F} = \cfrac{1}{r^2} \cdot \cfrac{\partial}{\partial r} (r^2 \cdot F_r) + \cfrac{1}{r \cdot \sin \vartheta} \Bigg[  \cfrac{\partial}{\partial \vartheta} (\sin \vartheta \cdot F_\vartheta) + \cfrac{\partial F_\varphi}{\partial \varphi} \Bigg] \)
                \item rot \(\vec{F} = \cfrac{1}{r \cdot \sin \varphi} \Bigg(  \cfrac{\partial}{\partial \vartheta}(\sin \vartheta \cdot F_\varphi) - \cfrac{\partial F_\vartheta}{\partial \varphi}  \Bigg)\vec{e}_r +
                \cfrac{1}{r} \Bigg(  \cfrac{1}{\sin \vartheta} \cdot \cfrac{\partial F_r}{\partial \varphi} - \cfrac{\partial}{\partial r}(r \cdot F_\varphi) \Bigg) \vec{e}_\vartheta + 
                \cfrac{1}{r} \Bigg(   \cfrac{\partial}{\partial r}(r \cdot F_\vartheta) - \cfrac{\partial F_r}{\partial \vartheta}  \Bigg)  \)
                \item \(\varDelta \phi = \cfrac{1}{r^2}\Bigg\{  \cfrac{\partial}{\partial r} \Bigg( r^2 \cdot \cfrac{\partial \phi}{\partial r}  \Bigg) + \cfrac{1}{\sin \vartheta} \cdot 
                \cfrac{\partial}{\partial \vartheta} \Bigg(  \sin \vartheta \cdot \cfrac{\partial \phi}{\partial \vartheta}  \Bigg)  + \cfrac{1}{\sin^2 \vartheta} \cdot \cfrac{\partial^2 \phi}{\partial \varphi^2}  \Bigg \}\)
            \end{itemize}
    \section{Linien- und Kurvenintegrale}
        Das Linienintegral ist angelehnt an der physikalischen Arbeit. Die Arbeit kann sich oft ändern, je nachdem wie der Weg gewählt wurde. Zum Beispiel kann eine Box diagonal im Raum verschoben werden
        oder entlang 2 Wänden also von x: 0 \(\rightarrow\) 1, y: 0 \(\rightarrow\) 1. Das Resultat ist dasselbe aber die Arbeit ist unterschiedlich. \\
        Der Weg wird allgemein als \(C\) beschrieben. Die Arbeit ist dann das Integral aus dem Skalarprodukt und dem Weg (vektorielle dargestellt) und einer Kraft/einem Kraftfeld. Die Integralgrenzen
        sind dann die Punkte die der Weg durchlaufen soll.
        \begin{equation*}
            \begin{aligned}
                & \text{1. Weg C bestimmen und als Vektor \(\vec{r}\) darstellen. } \\
                & \text{2. \(x\) mit \(t\) in \(\vec{r}\) substituieren und \(y\) dementsprechend anpassen} \\
                & \text{3. \(x(t)\) und \(y(t)\) in den Kraftvektor \(\vec{F}\) einsetzen.  } \\
                & \text{4. \(\vec{r(t)}\) ableiten } \\
                & \text{5. Das Integral bilden durch: } {\int\limits_C} \vec{F} \cdot d\vec{r} = {\int\limits_{t1}}^{t2} \vec{F} \cdot \dot{\vec{r}} \; dt
            \end{aligned}
        \end{equation*}
        Analog gilt dasselbe für Vektoren mit drei Komponenten. Wenn rot\((\vec{F} = 0)\)\footnote{rot(grad(\(\phi)) = \vec{0} \)} gilt (Vektorfeld ist konservativ),
        dann ist das Linienintegral wegunabhängig dh. der Weg kann frei gewählt werden.
    \section{Oberflächenintegral}
        Das Oberflächenintegral wird auch oft Flussintegral genannt, da man durch das Oberflächenintegral oft den Fluss durch eine Fläche bestimmen kann. Dafür muss eben der Fluss (Vektorfedl \(\vec{F}\)) 
        und die Fläche (\(\vec{v}\)) bestimmt sein. Für \(\vec{v}\) bietet sich meistens oft an die \(z\)-Komponente durch \(x\) und \(y\) (\(r, \varphi\) etc.) darzustellen. \(\vec{v}\) wird nun nach 
        \(x\) und \(y\) abgeleitet und aus dem Kreuzproukt der beiden Vektoren entsteht dann der Normalenvektor \(\vec{n}\). Das Skalarprodukt aus \(\vec{F}\) und \(\vec{v}\) wird nun doppelt integriert
        nach \(x\) und \(y\) (die Grenzen müssen gegebenenfalls durch die Anfangsfläche erst bestimmt werden). Das Ergebnis ist dann das Oberflächenintegral:
        \begin{equation*}
            \begin{aligned}
                & \text{1. Fläche und Vektorfeld bestimmen und parametriesieren (polar, kartesische, etc.)} \\
                & \text{2. Fläche als Vektor \(\vec{v}\) darstellen und \(z\) durch \(x\) und \(y\) darstellen} \\
                & \text{3. \(\vec{v}\) nach \(x\) und \(y\) ableiten und das Kreuzprodukt bilden} \\
                & \vec{n} = \cfrac{\vec{v}}{dx} \times \cfrac{\vec{v}}{dy} \\
                & \text{4. Skalarprodukt aus \(\vec{F}\) und \(\vec{n}\) bilden} \\
                & \text{5. Ergebnis doppelt nach dx und dy integrieren.}\footnote{Integrat<ionsgrenzen müssen ggf. berechnet werden} \\
                & \iint\limits_{M} \vec{F} \cdot \vec{n} \; dxdy
            \end{aligned}
        \end{equation*}
        Es kann auch vorkommen, dass zum Beispiel das Oberflächenintegral von einer geschlossenen Halbkugeln berechnet werden soll. Hier muss die 'Bodenfläche' extra berechnet werden und zur 
        Mantelfläche später dazuaddiert werden. 
\end{document}