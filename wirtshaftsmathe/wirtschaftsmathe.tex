\documentclass[a4paper]{scrartcl}
\pdfinfoomitdate=1
\pdftrailerid{}
\author{}
\date{WS 20/21}
\title{Grundlagen der Informatik}
\usepackage[utf8]{inputenc} % use utf8 file encoding for TeX sources
\usepackage[T1]{fontenc}    % avoid garbled Unicode text in pdf
\usepackage[german]{babel}
\usepackage{amsmath}
\usepackage{amssymb}
\usepackage{enumitem}
\begin{document}
    \maketitle
    \newpage
    \tableofcontents
    \newpage

    \section{Lineare Optimierung}
    Unter Lineare Optimierung versteht man, dass eine bestimmte Funktion durch einstellen bestimmter 'Parameter' möglichst groß (oft für Gewinn benutzt) oder möglichst klein (für Verluste) wird.
    Dafür wird die lineare Zielfunktion: 
    \begin{equation*}
        g(\vec{x}) = g_0 + g_1x_1 \ldots g_nx_n
    \end{equation*}
    hergenommen. Zum Beispiel ist \(g_i\) dann der Preis für die Herstellung eines Produkts und \(x_i\) dann die Anzahl wie oft das Produkt hergestellt werden soll und \(x_i\) sind eben die 
    Parameter die eingestellt werden sollen. Dazu sind idR. Randbedingungen gegeben, wie '\(x_1\) darf höchstens doppelt so oft hergestellt werden wie \(x_2\)'. Dann gibt es noch triviale 
    Randbedingungen. Da man nicht negativ produzieren kann gilt meistens auch \(x_i \ge 0\). Somit erhält man einige Ungleichungen und Gleichung. Zuerst schriebt man alle (Un)Gleichungen 
    untereinander auf und dreht jedes Ungleichungszeichen (Multiplikation mit -1) um damit alle Ungleichungszeichen in die selbe Richtung zeigen. Nun versucht man Variablen zu eliminieren
    indem man de GLeichungen die man hat nach einer Variable umstellt und dann diese Variable in den Ungleichungen ersetzt. Beispiel:
    \begin{equation*}
        x_1 + x_2 = 0 \Rightarrow x_1 = -x_2    
    \end{equation*}
    \(x_1\) kan in diesem Beispiel mit \(-x_2\) in den Ungleichungen ersetzt werden. Wenn man die ganzen Gleichungen in ein Diagramm einzeichnet erhält man ein Vieleck mit Kanten die eben 
    von den Ungleichungen dargestellt werden. Man kann jetzt jeden Schnittpunkt anschauen, die Werte berechnen die herauskommen wenn man die x-Koordinaten in die Gewinngleichung einsetzt und
    vergleicht wie jeden Wert und sucht sich den höchsten/niedrigsten Wert als Lösung heraus. Bei vielen Parametern kann das schnell viele Berechnungen benötigen. Man kann auch das Diagramm wieder
    hernehmen. Hier sucht man sich einen geeigneten Startpunkt (eine Punkt an einer Kante) und schaut ob man ob die Kante ansteigt oder waagrecht verläuft und folgt dieser Kante bis man 
    einen Schnittpunkt findet. Hier wird wiederum geschaut ob es eine Kante gibt die ansteigt oder waagrecht verläuft. Dies wird wiederholt bis das nicht mehr möglich ist. Der Letzt gefundene
    Schnittpunkt, welcher noch das Vieleck umspannt, erhält dann die Parameter für die Lösung. 
    \subsection{Simplexalgorihmus}
    Da man schwer vier, fünf oder sechs Dimensionen zeichnen kann gibt es den Simplexalgorihmus welcher den Optimalwert rechnerisch ermitteln kann. Hier muss zuerst ein Tablot erstellt werden.
    Die Form des Tablots ist wie folgt:
    \(
        \begin{pmatrix}
            \begin{matrix}
                 &   & \\
                 & A & \\
                 &   & 
            \end{matrix}
            \begin{matrix}
               & \\ & \vec{k} \\ &   
            \end{matrix} \\

            \hline
            \begin{matrix}
                &\vec{g} & &  G \\             
            \end{matrix}   
        \end{pmatrix}
    \)
    Wobei Die Matrix A Die Ungleichungen ind Matrixform ist, \(\vec{g} = G\) eine Lösung ist von der man weiß, dass sie die Gewinnfunktion erfüllt und sich \(\vec{k} = A\vec{u} - \vec{b}\). Dabei ist
    \(\vec{u}\) auch der Vektor der die Gewinnfunktion erfüllt. 

    \section{Finanzmathematik}
        \subsection{Folgen und Reihen}
        Eine Zahlenfolge ist eine Abbildung, welche wie folgt definiert ist:
        \begin{equation*}
            a = \mathbb{N} \rightarrow \mathbb{R}
        \end{equation*}
        Anstatt \(a(i)\) schriebt man \(a_i\) für jedes einzelne Glied der Folge. 
        Eine Reihe ist eine Aufsummierung aller Folgeglieder und wird unendliche Reihe genannt wenn unendlich viele Glieder aufsummiert werden bzw. endliche Reihe genannt wenn bis zu einem 
        bestimmten index aufsummiert wird. 
        
        \begin{equation*}
            \text{arithmetische Zahlenfolge: } a_n = a_1 + (n - 1)d, d \epsilon \mathbb{R} \text{ fest} 
        \end{equation*}
        \begin{equation*}
            \text{endliche arithmetische Reihe: } \frac{N(2a_1+(N+1)d)}{2}
        \end{equation*}
        \begin{equation*}
            \text{geometrische Zahlenfolge: } a_n = a_1 \cdot q^{n-1}
        \end{equation*}
        \begin{equation*}
            \text{geometrische Reihe: } a_1 \frac{q^N - 1}{q - 1}
        \end{equation*}

        \subsection{Einfache Verzinsung}
        Sollen Zinsen nicht miteinander Verzinst werden (Zinsen werden auf ein anderes Konto ausgezahlt) spricht man von einfacher Verzinsung. Es gilt folgenden Formel:
        \begin{equation*}
            K_n = K_0(1 + n \cdot i)
        \end{equation*}
        Wobei \(K_n\) das Kapitale nach \(n\) Zinsperioden darstellt. \(K_0\) ist dabei das Startkapital, \(i\) die Zinsrate\footnote{\(i \cdot 100 \equiv Zinsrate in Prozent\)} und \(n\)
        eben die Laufzeit in Zinsperioden (meistens Monate oder Jahre).
            \subsubsection{Verzinsung mit Zinseszins}
            Hier werden die Zinsen mitverzinst (Zinsen werden aufs selbe Konto ausgezahlt). Die Formel lautet: 
            \begin{equation*}
                K_n = K_0(1+i)^n
            \end{equation*} 

            \subsubsection{Stetige Verzinsung}
            Unter stetiger Verzinsung versteht man, dass die Zinsen nicht jeden Monat oder jeden Tag angerechnet werden sondern zu jedem Augenblick. Dafür wird folgende Formel werwendet: 
            \begin{equation*}
                K_n = K_0 e^{i \cdot n}
            \end{equation*}

            \subsubsection{Gemischte Verzinsung}
            Wenn man als Zinsperiode ein ganzes Jahr hat aber man die Verzinsung mitten in der Zinsperiode 'abbrechen' will wird folgende Formel benutzt:
            \begin{equation*}
                K_n = K_0(1 + i)^{\lfloor n \rfloor} 
                \cdot (1 + i (n - \lfloor n \rfloor)\footnote{Die Klammern sind Gaußsche Rundungsklammern. Sind sie wie hier nach unten gerichtet, wird abgerundet und wenn sie nach oben gerichtet sind, wird aufgerundet})
            \end{equation*}

            \subsubsection{Freigeld}
            Die Idee hinter Freigeld ist dass Geld nicht wir Lebensmittel oä. verderben kann. Um das zu korrigieren kann man die Formel für einfache Verzinsung (wenn man Bargeld hat) bzw. 
            die Formel für den Zinseszins hernehmen (Girokonto). Hier nimmt man als Zinsrate einfach den Negativwert. 
        \subsection{Investitions- und Finanzrechnung}
        Die Investitionsrechnung wird verwendet um zu bestimmen ob sich eine Investitions im Laufe der Zeit lohnen wird. Zuerst zieht man die Kosten der Investitions vom Kapital ab.
        Gegenfalls schaut man ob es periodische Zahlungen gibt die man leisten muss (Wartungsarbeiten, Kreditszinsen etc.) und zieht diese auch ab. Man schätz nun den Gewinn den die Investition im Monat/Jahr etc.
        macht, zieht noch eventuelle Zahlungen ab und erhält den Monats/Jahresgewinn. Dies macht man für alle Monate für die eine Schätzung machbar ist, verzinst diese dementsprechend und vergleicht den 
        Gesamtgewinn mit dem Gewinn den man erhält wenn man nicht investiert mit dem über dem selben verzinsten Kapital, das man hat wenn man nicht investiert.

        \subsection{Rentenrechnung}
        Unter einer Rente versteht man im Allgemeinem eine regelmäßige Ein- oder Auszahlung. Es wird zwischen auch noch eine Unterscheidung getroffen wenn die Zinsperiode der Rentenperiode gleicht
        (man zahl jeden Monat einen gewissen Betrag aufs Sparbuch und bekommt jeden Monat auch Zinsen auf das Geld) und wenn die  Rentenperiode kleiner der Zinsperiode ist (man zahlt monatlich ein 
        bekommt aber nur jährlich Zinsen).
            \subsubsection{Zinsperiode = Ratenperiode}
                Die allgemeine Formel lautet: 
                \begin{equation*}
                    R_n = r \cdot q \cdot \frac{q^n - 1}{q-1}
                \end{equation*}
                \(r\) ist die Ratenzahlung, \(n\) wieder die Anzahl der Ratenperioden, \(q\) berechnet sich durch \(q = 1 +i\), wobei \(i\) wieder der Zins ist. Die Formel gilt für vorschüssige Ratenzahlungen und nachschüssige Zinszahlungen.
                Für die nachschüssige Ratenzahlung gilt: 
                \begin{equation*}
                    R_n = r \cdot \frac{q^n - 1}{q - 1}
                \end{equation*}
            %TODO umformungen in eine Tabelle schreiben
            \subsubsection{Rentenperiode < Zinsperiode}
                Wenn die Rentenperiode kleiner der Zinsperiode ist muss man nur die Ratenzahlung \(r\) aus der Formel für die nachschüssige Ratenzahlung ändern. Es gilt für das neue \(r_e\):
                \begin{equation*}
                    r_e = r \cdot (m + \frac{i}{2} \cdot (m + 1)) \Rightarrow R_n = r_e \frac{q^n-1}{q-1} = r \cdot (m + \frac{i}{2} \cdot (m + 1)) \cdot \frac{q^n-1}{q-1}
                \end{equation*}
                Wobei das \(m\) die Anzahl der Ratenzahlungen pro Zinsperiode ist.
        
        \subsection{Tilgungsrechnung}
        Bei der Tilgungsrechnung geht es idR. umd die Rückzahlung eines Kredit (bzw. allgemein Schulden). Die Annuität bezeichnet damit den Betrag, der periodisch zurückgezahlt werden muss.
        Die Annuität setzt sich wiederum aus der Tilgungsrate (Betrag der zurückgezahlt werden soll) und dem Zinsbestandteil (Zinsen die pro Periode von den Restschulden dazukommen).
            \subsubsection{jährliche Ratentilgung}
            Soll die Tilgungsrate jedes Jahr (Periode) gleich bleiben ergibt sich folgende Formel:
            \begin{equation*}
                T = \frac{S}{n}
            \end{equation*}
            Wobei \(T\) die Gesamtschuld ist und \(n\) die Anzahl der Jahre (perioden) der Kreditlaufzeit.\\
            Will man die Zinsrate für ein bestimmtes Jahr (hier \(r\)) ausrechnen gilt folgende Formel: 
            \begin{equation*}
                Z_r = T \cdot (n - r + 1) \cdot i
            \end{equation*} 
            \(i\) ist dabei wieder der Zinssatz, n Die Kreditlaufzeit und \(T\) die Tilgungsrate. \\
            Die Annuität ergibt sich dann wenn man die beiden speziellen Formeln nun in die allgemeine Formel von oben einsetzt und umformt: 
            \begin{equation*}
                A = T \cdot (1 + (n - r + 1)i)
            \end{equation*}
            Für die Annuitätentilgung gilt folgende Formel: 
            \begin{equation*}
                A = S \cdot q^n \cfrac{q - 1}{q^n - 1}
            \end{equation*}
            \subsubsection{unterjährliche Ratentilgung}
            Soll mehrmals im Jahr getilgt werden muss der jährliche Zinssatz durch die Anzahl der Tilgungen im Jahr geteilt werden. Weiter beschreibt das \(n\) nicht mehr das ahr sondern die Anzahl
            der Zinsperioden.  

    \section{Deskriptive Statistik}
        \subsection{Stichproben}
            Bei der Statistik versucht man allgemeine Aussagen über einen bestimmten Datensatz zu erheben. Bei Millionen von Daten ist es oft nicht sinnvoll alle 
            Daten einzeln zu betrachten. Die Datenmenge wird auch als Beobachtungsmenge (B) bezeichnet. Jedem Element aus B wird dann ein bestimmtes Merkmal zugeordnet.
            Die Merkmale können disktret\footnote{\(\mathbb{N}_0\) also alles was abzählbar ist wie Gehalt oder Anzahl von Besuchern etc.} oder stetig\footnote{Intervalle in \(\mathbb{R}\), beliebig genau ermittelbare Werte
            wie Zeit Länge} sein. Weiter ist Urliste eine Teilmenge aus B mit den zugehörigen Merkmalswerten. Die Merkmalswerte aus der Urliste bilden dann die Stichprobe, welche 
            geordnet oder ungeordnet sein kann. Der Stichprobenumfang (n) ist dann die Kardinalität der Stichprobenmenge.
        \subsection{Häufigkeiten, empirische Verteilungsfunktion}
            Bei Häufigkeiten wird zwischen der absoluten und relativen Häufigkeiten unterschieden. Dafür nimmt man die zuerst die Funktion \(g(x)\) her, welche zählt wie oft
            ein Element \(x\) in der Stichprobe vorkommt. Dies ist dann die absolute Häufigkeit. Teilt man die absolute Häufigkeit durch \(n\)\footnote{Stichprobenumfang}.
            \begin{equation*}
                \begin{aligned}
                    & \text{absolute Häufigkeit:} & g(x) \neq 0 \Leftrightarrow x \in \{ x_1,x_2,..., x_n\} \subset \mathbb{R} \; (\mathbb{N}) \\
                    & \text{relative Häufigkeit:} & h(x) = \frac{1}{n} g(x) 
                \end{aligned}
            \end{equation*}    
            Die Summenhäufigkeit (\(G(x)\) summiert alle Werte aus der Stichprobe bis einschließlich \(x\). Die relative Häufigkeit nimmt die Summenhäufigkeit und teilt
            diese durch den Stichprobenumfang.
            \begin{equation*}
                \begin{aligned}
                    & \text{Summenhäufigkeit} & G(x_i) = x_1 + x_2 + x_3 + \ldots x_i & | \; x_i < x_n  \\
                    & \text{relative Summenhäufigkeit:} & H(x) = \frac{1}{n} G(x) &  
                \end{aligned}
            \end{equation*}  
            Die relative Summenhäufigkeit wird auch die empirische Verteilfunktion genannt. Durch die Summenhäufigkeit werden Häufigkeiten zusammengefasst und man kann
            allgemeinere Aussagen für die zusammengefassten Häufigkeiten machen.
        \subsection{Klasseneinteilung}
            Oft ist es sinnvoll Stichprobenwerte in Klassen einzuteilen. Dafür wählt man zuerst die Anzahl an Klassen die man haben möchte (hie: \(l\)). Nun teilt man die
            den Stichprobenumfang durch \(l\). Dadurch erhält man die Klassenbreite \(w\) (ggf. muss gerundet werden). Die Klassenbreite gibt an wie viele Einzelemenet zusammengefasst werden
            soll (durch Addition). Hat man alle Klassen zusammengefasst kann man nun allgemeinere Aussagen über bestimmte Teile der Stichprobe machen (Aussagen über den Verdienst von 
            Personen die weniger als 20h, zwischen 20h und 40h und mehr als 40h die Woche arbeiten als Beispiel). \\
            Für die Klasseneinteilung sollte man sich an diese Regeln halten: 
            \begin{itemize}
                \item Möglichst gĺeich große Klassen
                \item 5 \(\le\) \(l\) \footnote{Anzahl der Klassen} \(\le\) 25
                \item l \(\approx\) \(\sqrt{n}\)
                \item Klassengrenzen solten ganze Zahlen sein 
                \item Klassen sollten den Wertebereich der Stichprobe möglichst gut abdecken
            \end{itemize}  
        \subsection{Statistische Parameter}
            Desto allgemeiner man eine Stichprobe behandelt, desto übersichtlicher  aber auch desto ungenauer werden die Informationen.
            Es gibt statistische Parameter welche die Stichprobe in eine einzelne Zahl zusammenfassen: 
            \begin{equation*}
                \begin{aligned}
                    & \text{arithmetisches Mittel/Stichprobenmitte: } & \overline{x} = \frac{1}{n} \sum_{i=1}^n x_i \\
                    & \text{Median für n = ungerade:} & \tilde{x} =  x_{(\frac{n+1}{2})}  \\
                    & \text{Median für n = gerade: } & \tilde{x} = \frac{1}{2} (x_{\frac{n}{2}} + x_{\frac{n}{2} + 1}) \\
                    & \text{Spannweite: } & v := x_{(n)} - x_{(1)} \\
                    & \text{Varianz: } & s_x^2 = \frac{1}{n - 1} \sum_{i=1}^n(x_i - \overline{x})^2 \\
                    & \text{Streuung/Standardabweichung: } & s_x = \sqrt{s_x^2} = \sqrt{\frac{1}{n - 1} \sum_{i=1}^n(x_i - \overline{x})^2 }  
                \end{aligned}
            \end{equation*}
        \subsection{Zweidimensionale Stichproben}
            Eine Zweidimensionale Stichprobe ist wie eine normale Stichprobe nur dass da jetzt noch ein weiteres Merkmal auftritt, wodurch Zahlenpaare entstehen 
            (zB. Anstatt nur Einkommen als Merkmal wird jetzt Einkommen und Steuern untersuch). Für zweidimensionale Stichproben gelten (mit kleinen Änderungen) die
            selben Definitionen:
            \begin{equation*}
                \begin{aligned}
                    & \text{Stichenprobenumfang: } & \text{n = Anzahl der (x,y) Zahlenpaare} \\
                    & \text{absolute Häufigkeit: } & g(x,y) \neq 0 \Leftrightarrow x,y \in \{ (x_1,y_1),(x_2,y_2),..., (x_n,y_n)\} \\
                    & \text{relative Häufigkeit: } & h(x,y) = \frac{1}{n} g(x,y) \\
                    & \text{Summenhäufigkeit: } & G(x,y) = \sum_{\{ (a,b): a\leq x, b\leq y \}} g(a,b) \\
                    & \text{empirische Verteilungsfunktion: } & H(x,y) = \sum_{\{ (a,b): a\leq x, b\leq y \}} h(a,b) \\
                    & \text{Klassenhäufigkeiten: } & \tilde{h_{ij}} = \sum_{x \in KX_i, y \in KY_j} = h(x,y) \\
                    & \text{Randhäufikeiten für ein festes x/y: } & h_{x/y} (x/y) = \sum_{x/y \in \mathbb{R}} h(x,y) \footnote{Mit x/y ist x oder y gemeint} \\
               \end{aligned}
            \end{equation*}
            
            \subsection{Lineare Regression}
                Zeichnet man die Stichprobenwerte in ein Diagramm ein erhält man einige Punkt in einem Diagramm. Mit der linearen Regression sucht man eine Funktion, welche ungefähr
                der Richtung der Punkten entspricht. Für die Funktion \(y = ax + b\) müssen \(a\) und \(b\) so gewählt werden, damit der Abstand von der Geraden zu den Punkten möglichst klein wird.
                Die einzelnen Abstände \(\Delta \) werden dann quadriert und danach zusammenaddiert. \(a\) und \(b\) sollen so gewählt werden, dass \(\Delta ^2\) möglichst klein wird.
                Diese Methode zur Ermittlung der Regressionsgeraden heißt "Methode der kleinsten Quadrate". Die Kovarianz bei zweidimensionalen Stichproben wird wie folgt ermittelt:
                \begin{equation*}
                   s_{xy}= \frac{1}{n - 1} \sum_{i=1}^n (x_i - \overline{x})(y_i - \overline{y}) = \frac{1}{n - 1}\sum_{i=1}^n x_i y_i - n \overline{xy}
                \end{equation*}
                Teilt man die Kovarianz durch die Standardabweichung von jeweils \(x\) und \(y\) erhält man den empirischen Korrelationskoeffizienten:
                \begin{equation*}
                    r_{xy} = \frac{s_{xy}}{s_x \cdot s_y}
                \end{equation*}
                Weiter gilt für den Korrelationskoeffizienten \(r_{xy}\): 
                \begin{itemize}
                    \item \(-1 \leq r_{xy} \leq 1\)
                    \item Wenn \(r_{xy} = 1\) oder \(r_{xy} = -1\) liegen die Messpunkte direkt auf einer Geraden
                    \item Wenn \(r_{xy} = 0\) dann haben die Merkmale \(X\) und \(Y\) keinen Zusammenhang
                    \item Gibt es mehr als 2 Merkmale kann man selber Paare bilden und sieht dadurch wie stark Merkmale direkt miteinander zusammenhängen 
                \end{itemize} 

    \section{Kombinatorik}
        \subsection{Einführung}
            Bei der Kombinatorik unterschiedet man zwischen der Permutation, Variation und Kombination und in diesen 3 Gruppen wird nochmals unterschieden ob Wiederholungen auftreten 
            können oder nicht. 
        \subsection{Permutation}
            Eine Anordnung von n unterscheidbaren Objekten (es können nicht zwei oder mehr gleiche Objekte auftreten) heißt Permutation ohne Wiederholung (\(P_n\))
            \begin{equation*}
                P_n = n!
            \end{equation*} 
            Wobei \(n\) die Anzahl der ausgewählten Objekte ist. \\
            Bei der Permutation mit Wiederholung können verschiedene Sorten auftreten (Sorten sind Mengen mit gleichen Objekten). Für \(n\) Objekte und \(k\) Sorten gilt:
            \begin{equation*}
                \begin{aligned}
                    & n = n_1 + n_2 + \ldots + n_k & | \text{ \(n_1,n_2\), etc sind die Sorten} \\
                    & P_n^{(n_1,n_2, \ldots ,n_k)} = \frac{n!}{n_1! \cdot n_2! \cdot ... \cdot n_k!}
                \end{aligned}
            \end{equation*} 
        \subsection{Variation}
            Die Variation is eine Anordnung von \(k\) Elementen aus einer Menge aus \(n\) Sorten (\(n \geq k\)). Wenn ein Objekt mehrfach vorkommen kann spricht man 
            von Variation mit Wiederholung (\({V_{w_n}}^{(k)}\)) ansonsten spricht man von einer Variation ohne Wiederholung (\({V_{n}}^{(k)}\)). Es gilt:
            \begin{equation*}
                \begin{aligned}
                    & {V_{n}}^{(k)} = n(n-1)(n-2) \ldots (n-k+1) = \frac{n!}{(n-k)!} \\
                    & {V_{w_n}}^{(k)} = n^k 
                \end{aligned}
            \end{equation*}
        \subsection{Kombinationen}
            Bei der Kombination ist im Vergleich zu Variation die Reihenfolge der Objekte nicht wichtig. Sind alle Objekte verschieden spricht man 
            von einer Kombination ohne Wiederholung (\({C_n}^{(k)}\)). Es gilt dann:
            \begin{equation*}
                {C_n}^{(k)} = \begin{pmatrix}
                                n \\ k 
                              \end{pmatrix}
                                                  = \frac{{V_n}^{(k)}}{k!}
            \end{equation*}
            Wenn aus einer Menge von Objekten \(n\) Sorten existieren und \(k\) Objekte ausgewählt werden und man dazu noch die Reihenfolge nicht berücksichtigt 
            spricht man von einer Kombination mit Wiederholung (\( {C_{w_n}}^{(k)} \)). Es gilt:
            \begin{equation*}
                {C_{w_n}}^{(k)} = \frac{(n-1+k)!}{k!(n-1)!} = \begin{pmatrix} n-1 + k \\ k  \end{pmatrix}
            \end{equation*}

    \section{Wahrscheinlichkeitstheorie}
        \subsection{Zufallsexperimente}
            Ein Zufallsexperimente ist ein beliebig wiederholbarer Versuch, bei dem eine bestimmte Menge als Ausgang möglich ist, die auch bei jeder Wiederholung gleich bleiben muss.
            Der Ausgang kann dabei auch nicht vorhergesagt werden.
            \begin{itemize}
                \item Jeder einzelnen Ausgang heißt Ergebnis (oa. Elementarereignis) und wird mit \(\omega\) bezeichnet.
                \item Die Menge aller Ereignisse heißt \(\Omega\)
                \item Eine Teilmenge von \(\Omega\) heißt Ereignis
                \item Ist \(A\) eine Teilmenge von \(\Omega\) so ist \(\overline{A}\) das komplementäre Ereignis zu \(A\). \(\overline{A}\) beinhält alle Ergebnisse die nicht in \(A\)
                    vorkommen \(\Rightarrow \overline{A} = \Omega \setminus A\)
            \end{itemize}  
            TODO: Ereignisalgebra
        \subsection{Laplace Wahrscheinlichkeit}
            Zufallsexperimente haben endlich viele Ereignisse. Um jetzt die Wahrscheinlichkeit eines bestimmten Ereignis mit anderen einfacher vergleichen zu können
            teilt man die Wahrscheinlichkeit \(A\) durch die Anzahl der Elemente aus \(\Omega\)\footnote{Bei Laplace wird angenommen dass alle Ereignisse die selbe Wahrscheinlichkeit haben}.
            Die Wahrscheinlichkeit ist jetzt in einem Intervall zwischen \(1\) und \(0\). Es gilt:
            \begin{equation*}
                p(A) = \frac{|A|}{n} = \frac{\text{Anzahl der Elemente aus }A}{\text{Anzahl der Elemente aus } \Omega}
            \end{equation*}
        \subsection{Axiome von Kolmogorow}
            Möchte man Zufallsexperimente durchführen, wobei die Ergebnisse verschiedene Wahrscheinlichkeiten haben, könnte man einen Versuch sehr oft durchführen um eine Näherung 
            einer Wahrscheinlichkeit zu bekommen. Dies kann oft nicht durchgeführt werden, weshalb man Eigenschaften definiert hat, fie Wahrscheinlichkeiten erfüllen müssen. \\
            Ein Zufallsexperiment hat der Ereignismenge \(\Omega\), wobei  \(A\) eine Ereignisalgebra auf \(\Omega\) ist. Nun heißt die Abbildung \(p: A \mathbb{R} \) Wahrscheinlichkeit oder
            auch Wahrscheinlichkeitsmaß wenn gilt:
            \begin{enumerate}
                \item 0 \(\le\) \(p(A)\) \(\le\) 1
                \item \(p(\Omega) = 1\)
                \item Sind die Ereignisse \(A_1,A_2,\cdots,A_n\) paarweise verschieden gilt \(\bigcup\limits_{i=1}^{n} A_i = \sum\limits_{i=1}^n p(A_i) \)
            \end{enumerate} 
            Für das Wahrscheinlichkeitsmaß gilt außerdem: 
            \begin{enumerate}[label=\alph*)]
                \item \(p(\{ \; \}) = 0\)
                \item \(A\) und \(B\) sind Ereignisse: \((A \cap B) = \{ \; \}  \Rightarrow p( A \cup B) = p(A) + p(B)\)
                \item \(p(A) + p(\overline{A}) = 1\)
                \item \(A\) und \(B\) sind Ereignisse: \(p(A \cup B) = p(A) + p(B) - p(A \cap B)\)
                \item \(A\) und \(B\) sind Ereignisse: \(A \subset B \Rightarrow p(A) \le p(B)\)
            \end{enumerate}
        \subsection{Bedingte Wahrscheinlichkeit}
            Bedingte Wahrscheinlichkeiten setzen vorraus dass eine bestimmtes Ereignis schon eingetroffen und man nun die Wahrscheinlichkeit der 'nächsten' Wahrscheinlichkeit berechnen möchte. \\
            Wenn \(A\) und \(B\) Ereignisse sind, \(p(B) \neq 0\) dann heißt 
            \begin{equation*}
                p_B(A) = p(A|B) := \cfrac{p (A \cap B)}{p(B)}
            \end{equation*} 
            bedingte Wahrscheinlichkeit von \(A\) unter der Bedingung von \(B\). Daraus folgt 
            \begin{equation*}
                p(A \cap B) = p(B) \cdot p(A|B) = p(A) \cdot p(B|A) 
            \end{equation*}
            Satz der totalen Wahrscheinlichkeit:\\
            \(B\) sein ein Ereignis. \(A_1,A_2,\cdots,A_3\) sind paarweise diskunkte Ereignisse die zusammen \(\Omega\) ergeben \(\Omega = \bigcup\limits_{i=1}^n A_i \), dann gilt: 
            \begin{equation*}
                p(B) = \sum\limits_{i=1}^n p(B|A_i) \cdot p(A_i)
            \end{equation*}
        \subsection{Satz von Bayes}
            \(A_1,A_2,\cdots,A_3\) sind paarweise disjunkte Ereignisse, \(\Omega = \bigcup\limits_{i=1}^n A_i\), \(B\) ist ein Ereignis und es gilt \(p(B) > 0\) außerdem ist \(i,j \; \epsilon \; \mathbb{N}\), so gilt:
            \begin{equation*}
                P(A_i|B) = \cfrac{p(B|A_i) \cdot A_i}{\sum\limits_{j=1}^n p(B|A_j) \cdot p(A_j)} %TODO
            \end{equation*}
        \subsection{Unabhängigkeit von Ereignissen}
            \(A, B\) seien Ereignisse, \(A\) ist Unabhängig von \(B\) wenn gilt:
            \begin{equation*}
                p(A) = p(A|B)
            \end{equation*}
            Daraus folgt, dass auch gelten muss: 
            \begin{equation*}
                p(B) = p(B|A)
            \end{equation*}
    \section{Zufallsvariable und statistische Verteilung}
        \subsection{Zufallsvariable}
            Eine Zufallsvariable ist eine Funktion die jedem Ereignis aus \(\Omega\) eine Zahl in \(\mathbb{R}\) zuordnet. Die Zufallsvariable \(X\) is somit eine Funktion und kein Wert.
            \begin{equation*}
                X: \Omega \rightarrow \mathbb{R}, x=X(\omega)
            \end{equation*}
            \(X\) heißt diskrete Zufallsvariable, wenn deren Wertebereich endlich oder abzählbar undendlich ist. \\
            Die Funktion  \(f(X): \mathbb{R} \rightarrow \mathbb{R}\) mit 
            \[f(X)=
                \begin{cases}
                    p_i & \text{wenn } x=x_i\\
                    0 & \text{für den Rest} 
                \end{cases}
            \] 
            heißt Wahrscheinlichkeitsfunktion bzw. Dichtefunktion der diskreten Zufallsvariablen \(X\). \\
            Die Verteilung von \(X\) ist definiert als 
            \begin{equation*}
                p_X(A) = \sum\limits_{x_i \varepsilon A} f_x(x_i)
            \end{equation*}
            wobei \(p_X = 2^{W_x} \rightarrow [0;1] \), also die Potenzmenge des Wertebereichs. Weiter ist die Verteilungsfunktion definiert als 
            \begin{equation*}
                F_X: \mathbb{R} \rightarrow \mathbb{R} \text{ mit } F_X (x) = p(X \le x) = \sum\limits_{x_i < X} f_X(x_i)
            \end{equation*}
            Erwartungswert:
            \begin{equation*}
                E(X) = \sum\limits_{x_i \epsilon W_X} x_i \cdot f_X (x_i)
            \end{equation*}
            Varianz:
            \begin{equation*}
                \sum\limits_{x_i \epsilon W_X} (x_i - E(X))^2 \cdot f_X(x_i)
            \end{equation*}
            Varianz und Erwartungswert sind linear, dh. wenn eine neue Zufallsvariable gemacht wird, die (wie folgt) abhängig von Konstanten ist (\(Z_n := a \cdot X + b\)) dann gilt
            \begin{equation*}
                \begin{aligned}
                    E(Z_n) = a \cdot E(X) + b & \text{ | für den Erwartungswert}\\
                    Var(z) = a^2 \cdot Var(X) & \text{ | für die Varianz}
                \end{aligned}
            \end{equation*}  

        \subsection{Spezielle disktret Verteilungen}
            \subsubsection{Gleichverteilung}
                Eine Verteilung mit der Dichte 
                \[f(x)=
                    \begin{cases}
                        \frac{1}{n} & \text{wenn } x \; \epsilon \; W_X (|W_X| = n)\\
                        0 & \text{für den Rest} 
                    \end{cases}
                \] 
                heißt Gleichverteilung. Für den Erwartungswert gilt 
                \begin{equation*}:
                    E(X) = \frac{1}{n} \sum\limits_{x\epsilon W_X} x
                \end{equation*}
                Ist  \(W_x = 1,2,3, \cdots ,n\) gilt: 
                \begin{equation*}
                    E(X) = \frac{n+1}{2}
                \end{equation*} 
            \subsubsection{Bernoulli-Verteilung}
                Wenn die Zufallsvariable nur 2 Werte liefern kann, spricht man von einer Bernoulli-Verteilung, es gilt:
                \begin{itemize}
                    \item Dichte: \[f(x)=
                                                        \begin{cases}
                                                            p     & \text{für } x = 1 \\
                                                            1 - p & \text{für } x = 0 \\
                                                            0     & \text{für den Rest} 
                                                        \end{cases}
                                                        \] 
                    \item Erwartungswert: \(E(X) = p \)
                    \item Varianz: \(V(X) = pq = p (1 - q)\) \footnote{\(p = p(X=1), q = p(X=0)\)} % TODO aus wikipedia, in VL nachschauen
                \end{itemize}
            \subsubsection{Binomialverteilung}
                Die Binomialverteilung ist die Bernoulli-Verteilung \(n\) Mal durchgeführt, bzw. \(n\) Objekte aus denen \(x\) untersucht werden wenn
                nur eine Wahrscheinlichkeit \(p\) vorhanden ist. 
                \begin{itemize}
                    \item Dichte: \[f(x)=
                                                        \begin{cases}
                                                            \begin{pmatrix} n \\ x \end{pmatrix} p^x \cdot q^{n - x} & \text{für } x \; \epsilon \; \{0,1,2,\cdots, m\} \\
                                                            0     & \text{für den Rest} 
                                                        \end{cases}
                                                        \] 
                    \item Erwartungswert: \(E(X) = \sum\limits_{x =0}^n x \begin{pmatrix} n \\ x \end{pmatrix} \cdot p^x \cdot q^{n - x} = n \cdot p \)
                    \item Varianz: \(V(X) = \sum\limits_{x=0}^n (x - n \cdot p)^2 \cdot \begin{pmatrix} n \\ x \end{pmatrix} \cdot p^x \cdot q^{n - x} = n \cdot p \cdot q\) 
                \end{itemize}
                \(p\) ist dabei  die Wahrscheinlichkeit für's Eintreten des Ereignis. \(q\) ist definiert durch \(q = 1 - p\) und \(x\) ist die Anzahl wie oft das Ereignis eintreten soll.  
            \subsubsection{Geometrische Verteilung}
            Anzahl der Versuche die gebraucht werden bis der 1. Erfolg bei einem Bernoulli-Experiment eintritt. n ist dabei die Anzahl der Versuche (Wie hoch ist die 
            Wahrscheinlichkeit, dass das Ereignis bpojis zum n-ten Versuch eintritt).
            \begin{itemize}
                \item Dichte: \[f(x)=
                                                    \begin{cases}
                                                        q^np     & \text{für } n \; \epsilon \; \mathbb{N}_0 \\
                                                        0     & \text{für den Rest} 
                                                    \end{cases}
                                                    \] 
                \item Erwartungswert: \(E(X) = \cfrac{q}{p}\)
                \item Varianz: \(V(X) = \cfrac{q}{p^2}\) 
            \end{itemize}
            \subsubsection{Hypergeometrische Verteilung}
                Aus einer Grundmenge (Gesamtmenge) N werden erfüllen N Elemente eine bestimmte Eigenschaft (Aus 10 (N) Lampen sind 2 (M) defekt).
                Aus N werden nun n Elemente betrachtet. f gibt dann die Wahrscheinlichkeit für x auftauchende Elemnte an (Aus 10 Lampen sind 2 defekt,
                es werden 5(n) Lampen untersucht. Wie Wahrscheinlich ist es dass eine (x) aus den 5 Lampen kaputt ist).
                \begin{itemize}
                    \item Dichte: \(\cfrac{\begin{pmatrix} M \\ x \end{pmatrix} \cdot \begin{pmatrix} N - M \\ n - x \end{pmatrix}}{\begin{pmatrix} N \\ n \end{pmatrix}}\)
                    \item Erwartungswert: \(E(X) = n \cdot \cfrac{M}{N}\)
                    \item Varianz: \(V(X) = n \cfrac{M}{N} \Bigg(1 - \cfrac{M}{N} \Bigg) \cfrac{N - n}{N - 1} \)
                \end{itemize}
            \subsubsection{Poisson-Verteilung}
                Bei der Poisson-Verteilung ist der Erwartungswert \(\lambda\) oft schon gegeben, da man gerade ihn zur Berechnung für die Dichte benötigt. 
                Poisson eignet sich für Stichproben mit hohem Umfang (\(n = 100, 1000,\) etc) aber niedrigem Erwartungswert (\(\lambda = 2\)) Es gilt:
                \begin{itemize}
                    \item Dichte: \(f(x) = \cfrac{\lambda^k}{k!} \cdot e^\lambda\)
                    \item Erwartungswert: \(E(X) = \lambda\)
                    \item Varianz: \(V(X) = \lambda\)
                \end{itemize} 
                Dabei ist \(k\) zum Beispiel Anzahl der defekten Geräte.
            
        \subsection{stetige Zufallsvariable}
            Eine Zufallsvariable heißt stetig wenn \(W_X\) ein Intervall in \(\mathbb{R}\), die Dichtefunktion \(f_X\) gilt, dass sie nie negativ ist und und die Verteilfunktion 
            \(F_X(x)\) das Integral aus \(f_X\), also \(F_X(x) = \int\limits_{-\infty }^x f_X(u) \; du\). 
            \begin{itemize}
                \item Erwartungswert: \(E(X) = \int\limits_{-\infty}^{\infty} x \cdot (f_X(x)) \; dx \)
                \item Varianz: \(V(X) = \int\limits_{-\infty}^{\infty} (x - E(X))^2 \cdot f_x(x) \; dx = E(X^2) - E(X)^2 \)
            \end{itemize}
        \subsection{Spezielle stetige Verteilungen}
            \subsubsection{Gleichverteilung}
            \begin{itemize}
                \item Dichte: \[f(x) = f(x,a,b)=
                                \begin{cases}
                                    \frac{1}{b - a}     & \text{für } x \; \epsilon [a, b] \\
                                    0     & \text{für den Rest} 
                                \end{cases}
                            \] 
                \item Erwartungswert: \(E(X) = \cfrac{a+b}{2} \)
                \item Varianz: \(V(X) = \cfrac{(b-a)^2}{12} \)
            \end{itemize}
            \subsubsection{Normalverteilung}
                Die Normalverteilung ist die häufigste Verteilung in der Praxis.\\
                Dabei ist  \(W_X = \mathbb{R}, \mu \; \epsilon \; \mathbb{R}, \sigma \; \epsilon \; \mathbb{R}^+ \), somit gilt: 
                \begin{itemize}
                    \item Dichte: \( f(x, \mu, \sigma)  = \varphi(x, \mu, \sigma) = \cfrac{1}{\sqrt{2\pi} \sigma} \cdot e^{- \frac{(x - \mu)^2}{2 \sigma^2}} \) 
                    \item Erwartungswert: \(E(X) = \mu\)
                    \item Varianz: \(V(X) = \sigma^2\)
                    \item Standardnormalverteilung (\(\mu = 0, \sigma = 1\)):  \(\varPhi(x) = \varPhi(x,0,1) = \cfrac{1}{\sqrt{2\pi}} \int\limits_{-\infty}^xe^{\ - \frac{u^2}{2}} \; du\)
                \end{itemize}
                Umrechnung von \(X\) ins Standardnormalverteilung \(Z\): 
                \begin{equation*}
                    Z(X) = \cfrac{X - \mu}{\sigma}  = \cfrac{X - E(X)}{\sqrt{Var(x)}} 
                \end{equation*}
            \subsubsection{Exponentiallverteilung}
                Die Exponentialverteilung ist ein Model für die Lebensdauer für Produkte.
                \begin{itemize}
                    \item Dichte: \(f(x) = f(x, \lambda) = \lambda e^{-\lambda x}\) \text{ für \(x \ge 0\); ansonsten 0 für \(x < 0 \) }
                    \item Erwartungswert: \(E(X) = \cfrac{1}{\lambda} \)
                    \item Varianz: \(Var(X) =  \cfrac{1}{\lambda^2} \)
                \end{itemize}
            \subsection{Notizen}
                Das Intervall wird durch die Integration der Dichtefunktion ermittelt. \\
                Varianz und Erwartungswert haben den Wert 0 wenn das die Intervallgrenzen gleich sind (bzw. wenn nur ein Punkt gegeben ist und nicht zwei)
                \subsubsection*{Auswahltabelle}
                \begin{tabular}{|c|c|c|}
                    \hline
                    \; & diskret? & stetig? \\
                    \hline
                    anzahl? & Binomial/Hypergeometrische & Poisson \\
                    \hline
                    erstes Mal? & geometrisch & Exponential \\
                    \hline     
                \end{tabular}
    \section{Induktive Statistik und statistische Tests}
        \subsection{Schätzfunktion}
            Die Schätzfunktion \(\overline{X}\) gibt dann einen Wert für bestimmte Parameter(zB. Erwartungswert) Nun Werden verschiedene Stichproben erhoben
            und (unter Beachtung der vermuteten Verteilung) werden dann die Parameter (zB. Erwartungswert) ausgerechnet. Ist zB. \(\overline{X} = E(X)\) ist \(\overline{X}\)
            erwartungstreu für \(E(X)\). Allgemein ist jede Funktion \(U = g(X_1,X_2,...,X_n)\) eine Schätzfunktion für einen Parameter \(a\) einer Verteilung \(X\) (\(a\) darf aber nicht in \(X\) 
            vorkommen). \\
            Wenn ()\(E(U) = a\)) gilt, so ist die Schätzfunktion \(U)\) erwartungstreu.
        \subsection{Maximum Likelihood Methode}
            Für den Erwartungswert eignete sich das arithmetische Mittel über alle Kopien von \(X\) also aus allen \(X_i\).
            Da aber es aber viele Parameter und diese auch verschieden in anderen Verteilungen sein können, nutzt man die Maximum-Likelihood-Methode um bestimmte Parameter zu bestimmen.   
            Es gilt folgende Formel: 
            \begin{equation*}
                L (\vec{x},a) = \Pi_{i=1}^n f(x_i,a)
            \end{equation*}
            Das Ergebnis wird nach \(a\) abgeleitet und auf Nullstellen untersucht, da sich dort die Extremstellen befinden. Ggfs. muss noch einmal abgeleitet werden um den Höhepunkt 
            zu bekommen. Zu beachten ist, dass das \(a\) ein beliebiger Parameter einer Verteilung sein kann. Die verschiedenen \(x_i\) Werte sind Realisierungen von \(a\).
        \subsection{Hypothesentest}
            Bei einem Hypothesentest wird eine Aussage (Hypothese) auf ihre Wahrscheinlichkeit überprüft. Zum Beispiel Erwartungswert (Erwartungswert der Größe von Männern ist 1.72m). \\
            Dabei unterscheidet man zwischen einseitigen und symmetrischen Tests. Erwartungswert von Größen ist symmetrisch da wenn die Hypothese 'links oder rechts' abweicht verworfen wird.
            Einseitige Test schauen ob ein bestimmter Wert nicht überschritten worden ist ('Anzahl' defekter Smartphones bei eine Shop, '90\%' von Akkus gehen nach 2 Jahren kaputt). \\
            Ferner muss noch die Irrtumswahrscheinlichkeit \(\alpha\) festgelegt werden (oft 0.1, 0.05, 0.01). Eine Hypothese kann zu 70\% zutreffen, aber die restlichen 30\% bieten ein hohes 
            Risiko (Smartphones werden zurückgeschickt, Stichprobe vom Hersteller ist aber in Ordnung). \(\alpha\) ist daher meist klein. \(1 - \alpha\) ist die Sicherheitswahrscheinlichkeit. \\
            Symmetrisch Test haben zwei Grenzen (\(g_u, g_o\)). Die zwei Grenzbereiche werden links und rechts gleich (durch \(\alpha\)) aufgeteilt.
            \subsubsection*{Berechnung}
            \begin{enumerate}
                \item Eine Hypothese für \(X\) aufstellen
                \item Irrtumswahrscheinlichkeit \(\alpha\) festlegen, wenn nicht schon gegeben
                \item Grenzen berechnen. Für einseitige Test entweder Obergrenze oder Untergrenze, für symmetrische Tests Ober- und Untergrenze festlegen. 
                      Allgemein muss gelten: \(p(X < g_u) + p(X > g_o) = 1 \). Die Grenzen berechnet man aus der Umkehrung der Verteilfunktion:
                      \begin{itemize}
                          \item Einseitig: \(g_u = {F_X}^{-1} \Bigg( \alpha \Bigg) \) oder \(g_o = {F_X}^{-1} \Bigg(1 - \alpha \Bigg)\)
                          \item Symmetrisch: \(g_u = {F_X}^{-1} \Bigg( \cfrac{\alpha}{2} \Bigg) \) und \(g_o = {F_X}^{-1} \Bigg(1 - \cfrac{\alpha}{2} \Bigg) \)
                      \end{itemize} 
                \item Stichprobe wird durchgeführt und mit den Grenzen verglichen. Ist das beobachtete Merkmal in der Stichprobe wird die Hypothese angenommen. 
            \end{enumerate}   
            Bei einer Normalverteilung hilft es die Normalverteilung ind die Standardnormalverteilung tzu bringe, da man dadurch die Grenzwerte oft in einer Tabelle gegeben sind.
        \subsection{Zentraler Grenzwertsatz}
            Der zentrale Grenzwertsatz sagt aus, dass wenn \(n\) Zufallsvariablen (alle \(n\) Zufallsvariable sind Kopien voneinander) mit beliebiger Verteilung miteinander addiert werden (\(Y_n = X_1 + X_2 + ... + X_n\)),
            sich aus der Summe annähernd eine Normalverteilung bildet. Außerdem ist die Wahl des \(n\) wichtig (somit auch die Anzahl an Stichproben).
            Die Faustregel lautet:
            \begin{equation*}
                n = \cfrac{9}{p \cdot (1- p)}
            \end{equation*} 
            Für Varianz und Erwartungswert gilt noch folgendes:
            \begin{itemize}
                \item Erwartungswert: \(E(Y_n) = n \cdot E(X)\)
                \item Varianz: \(Var(Y_n) = n \cdot Var(X)\)
            \end{itemize} 
        \subsection{Varianztest}
            Um bei der Normalverteilung die Varianz überprüfen zu können, nutzt man die \(\chi^2\)-Verteilung. Diese setzt sich aus der Summe von \(m\) unabhängigen, standardnormalverteilten 
            Zufallsvariablen, welche alle jeweils einzeln quadriert werden:
            \begin{equation}
                X:= Z_1^2 + Z_2^2 + ... + Z_m^2
            \end{equation}
            Man spricht von einer \(\chi^2\)-Verteilung mit \(m\) Freiheitsgraden. Für Varianz und Erwartungswert gilt noch folgendenes:
            \begin{itemize}
                \item Erwartungswert: \(E(X) = m\)
                \item Varianz: \(Var(X) = 2m\)
            \end{itemize}
            Eine Zufallsvariable der Form
            \begin{equation*}
                Y:=\sum\limits_{i=1}^n \cfrac{(x_i - \overline{x})^2}{\sigma^2} = \cfrac{(n-1) s^2}{\sigma^2}
            \end{equation*}
            heißt \(\chi^2\)-verteilt mit \(m = n-1\) Freiheitsgraden. \(s^2\) ist dabei die empirische Varianz und \(\overline{x}\) der Mittelwerte der Stichprobe.\\
            Ablauf: Skript extra Seiten zu \(\chi^2\) ist schon gut zusammengefasst.
\end{document}