\documentclass[a4paper]{scrartcl}
\author{Me}
\title{Grundlagen der Informatik}
\usepackage[utf8]{inputenc} % use utf8 file encoding for TeX sources
\usepackage[T1]{fontenc}    % avoid garbled Unicode text in pdf
\usepackage[german]{babel}
\usepackage{amsmath}
\usepackage{amssymb}
\begin{document}
    \maketitle
    \newpage
    \tableofcontents
    \newpage

    \section{Lineare Optimierung}
    Unter Linerare Optimierung versteht man, dass eine bestimmte Funktion durch einstellen bestimmter 'Parameter' möglichst groß (oft für Gewinn benutzt) oder möglichst klein (für Verluste) wird.
    Dafür wird die öineare Zielfunktion: 
    \begin{equation*}
        g(\vec{x}) = g_0 + g_1x_1 \cdots g_nx_n
    \end{equation*}
    hergenommen. Zum Beispiel ist \(g_i\) dann der Preis für die Herstellung eines Produkts und \(x_i\) dann die Anzahl wie oft das Produkt hersgetellt werden soll und \(x_i\) sind eben die 
    Parameter die eingestellt werden sollen. Dazu sind idR. Randbedingungen gegeben, wie '\(x_1\) darf höchstens doppelt so oft hergestellt werden wie \(x_2\)'. Dann gibt es noch triviale 
    Randbedingungen. Da man nicht negativ produzieren kann gilt meistens auch \(x_i \ge 0\). Somit erhält man einige Ungleichungen und Gleichung. Zuerst schriebt man alle (Un)Hleichungen 
    untereinander auf und drhet jedes Ungleichungszeichen (Multiplikation mit -1) um damit alle Ungleichungszeichen in die selbe Richtung zeigen. Nun versucht man Variablen zu eliminieren
    indem man de GLeichungen die man hat nach einer Variable umstellt und dann diese Variable in den Ungleichungen ersetzt. Beispiel
    \begin{equation*}
        x_1 + x_2 = 0 \Rightarrow x_1 = -x_2    
    \end{equation*}
    \(x_1\) kan in diesem Beispiel mit \(-x_2\) in den Ungleichungen ersetzt werden. Wenn man die ganzen Gleichungen in ein Diagramm einzeichnet erhält man ein Vieleck mit Kanten die eben 
    von den Ungleichungen darsgetellt werden. Man kann jetzt jeden Schnittpunkt anschauen, die Werte berechnen die herauskommen wenn man die x-Koordianten in die Gewinngleichung einsetzt und
    vergleicht wie jeden Wert und sucht sich den höchsten/nidrigsten Wert als Lösung heraus. Bei vielen Parametern kann das schnell viele Berechnungen benötigen. Man kann auch das Diagramm wieder
    hernehmen. Hier sucht man sich einen geigneten Startpunkt (eine Punkt an einer Kante) und schaut ob man ob die Kante ansteigt oder waagrecht verläuft und folgt dieser Kante bis man 
    einen Schnittpunkt findet. Hier wird wiederum geschaut ob es eine Kante hǵibt die ansteigt oder waagrecht verläuft. Dies wird wiederholt bis das nicht mehr möglich ist. Der Letzt gefundene
    Schnittpunkt, welcher noch das Vieleck umspannt, erhält dann die Parameter für die Lösung. 
    \subsection{Simplexalgorihmus}
    Da man schwer vier, fünf oder sechs Dimensionen zeichnen kann gibt es den Simplexalgorihmus welcher den Optimalwert rechnerisch ermitteln kann. Hier muss zuerst ein Tabloet erstellt werden.
    Die Form des Tablots ist wie folgt:
    \(
        \begin{pmatrix}
            \begin{matrix}
                 &   & \\
                 & A & \\
                 &   & 
            \end{matrix}
            \begin{matrix}
               & \\ & \vec{k} \\ &   
            \end{matrix} \\

            \hline
            \begin{matrix}
                &\vec{g} & &  G \\             
            \end{matrix}   
        \end{pmatrix}
    \)
    Wobei Die Matrix A Die Ungleichungen ind Marixform ist, \(\vec{g} = G\) eine Lösung ist von der man weiß, dass sie die Gewinnfunktion erfüllt und sich \(\vec{k} = A\vec{u} - \vec{b}\). Dabei ist
    \(\vec{u}\) auch der Vektor der die Gewinnfunktion erfüllt. 

    \section{Finanzmathematik}
        \subsection{Folgen und Reihen}
        Eine Zahlenfolge ist eine Abbildung, welche wie folgt defineiert ist:
        \begin{equation*}
            a = \mathbb{N} \rightarrow \mathbb{R}
        \end{equation*}
        Anstatt \(a(i)\) schriebt man \(a_i\) für jedes einzelne Glied der Folge. 
        Eine Reihe ist eine Aufsummierung aller Folgeglieder und wird unendliche Reihe genannt wenn unendlich viele Glieder aufsummiert werden bzw. endliche Reihe genannt wenn bis zu einem 
        bestimmten index aufsummiert wird. 
        
        \begin{equation*}
            \text{arithmetische Zahlenfolge: } a_n = a_1 + (n - 1)d, d \epsilon \mathbb{R} \text{ fest} 
        \end{equation*}
        \begin{equation*}
            \text{endliche arithmetische Reihe: } \frac{N(2a_1+(N+1)d)}{2}
        \end{equation*}
        \begin{equation*}
            \text{geometrische Zahlenfolge: } a_n = a_1 \cdot q^{n-1}
        \end{equation*}
        \begin{equation*}
            \text{geometrische Reihe: } a_1 \frac{q^N - 1}{q - 1}
        \end{equation*}

        \subsection{Einfache Verzinsung}
        Sollen Zinsen nicht miteinander Verzinst werden (Zinsen werden auf ein anderes Konto ausgezahlt) spircht ma nvon einfacher Verzinsung. Es gilt folgenden Formel:
        \begin{equation*}
            K_n = K_0(1 + n \cdot i)
        \end{equation*}
        Wobei \(K_n\) das Kapitale nach \(n\) Zinsperioden dasrstellt. \(K_0\) ist dabei das Startkapital, \(i\) die Zinsrate\footnote{\(i \cdot 100 \equiv Zinsrate in Prozent\)} und \(n\)
        eben die Laufzeit in Zinsperioden (meistens Monate oder Jahre).
            \subsubsection{Verzinsung mit Zinseszins}
            Hier werden die Zinsen mitverzinst (Zinsen werden aufs selbe Konto ausgezahlt). Die Formel lautet: 
            \begin{equation*}
                K_n = K_0(1+i)^n
            \end{equation*} 

            \subsubsection{Stetige Verzinsung}
            Unter stetiger Verzinsung versteht man, dass die Zinsen nicht jeden Monat oder jeden Tag angerechnet werden sondern zu jedem Augenblick. Dafür wird folgende Fromel werwendet: 
            \begin{equation*}
                K_n = K_0 e^{i \cdot n}
            \end{equation*}

            \subsubsection{Gemischte Verzinsung}
            Wenn man als Zisnperiode ein gnazes Jahr hat aber man die Verzinsung miiten in der Zinsperiode 'abbrechen' will wird folgende Formel benutzt:
            \begin{equation*}
                K_n = K_0(1 + i)^{\lfloor n \rfloor} 
                \cdot (1 + i (n - \lfloor n \rfloor)\footnote{Die Klammern sind Gaussche Rundungsklammern. Sind sie wie hier nach unten gerichtet, wird abgerundet und wenn sie nach oben gerichtet sind, wird aufgerundet})
            \end{equation*}

            \subsubsection{Freigeld}
            Die Idee hinter Freigeld ist dass Geld nicht wir Lebensmittel oä. verderben kann. Um das zu korrigieren kann man die Formel für einfache Verzinsung (wenn man Bargeld hat) bzw. 
            die Formel für den Zinseszins hernehmen (Girokonto). Hier nimmt man als Zinsrate einfach den Negativwert. 
        \subsection{Investitions- und Finanzrechnung}
        Die Investionsrechnung wird verwendet um zu bestimmen ob sich eine Investions im Laufe der Zeit lohnen wird. Zuerst zieht man die Kosten der Investions vom Kapital ab.
        Gegenfalls schaut man ob es periodische Zahlungen gibt die man leisten muss (Wartungsarbeiten, Kreditszinsen etc.) und zieht diese auch ab. Man schätz nun den Gewinn den die Investion im Monat/Jahr etc.
        macht, zieht noch eventuelle Zahlungen ab und erhält den Monats/Jahresgewinn. Dies macht man für alle Monate für die eine Schätzung machbar ist, verzinst diese dementsprechend und vergleicht den 
        Gesamtgewinn mit dem Gewinn den man erhält wenn man nicht investiert mit dem über dem selben verzinsten Kapital, das man hat wenn man nicht investiert.

        \subsection{Rentenrechnung}
        Unter einer Renter versteht man im Allgemeinem eine regelmäßige Ein- oder Auszahlung. Es wird zwischen auch noch eine Unterschiedung getroffen wenn die Zinsperiode der Rentenperiode gleicht
        (man zahl jeden Monat einen gewissen Betrag aufs Sparbuch und bekommt jeden Monat auch Zinsen auf das Geld) und wenn die  Rentenperiode kleiner der Zinsperiode ist (man zahlt monatlich ein 
        bekommt aber nur jährlich Zinsen).
            \subsubsection{Zinsperiode = Ratenperiode}
                Die allgemeine Formel lautet: 
                \begin{equation*}
                    R_n = r \cdot q \cdot \frac{q^n - 1}{q-1}
                \end{equation*}
                \(r\) ist die Ratenzahlung, \(n\) wieder die Anzhal der Ratenperioden, \(q\) berechnet sich durch \(q = 1 +i\), wobei \(i\) wieder der Zins ist. Die Formel gilt für vorschüssige Ratenzahlungen und nachschüssige Zinszahlungen.
                Für die nachschüssige Ratenzahlung gilt: 
                \begin{equation*}
                    R_n = r \cdot \frac{q^n - 1}{q - 1}
                \end{equation*}
            %TODO umformungen in eine Tabelle schreiben
            \subsubsection{Rentenperiode < Zinsperiode}
                Wenn die Rentenperiode kleiner der Zinsperiode ist muss man nur die Ratenzahlung \(r\) aus der Formel für die nachschüssige Ratenzahlung ändern. Es gilt für das neue \(r_e\):
                \begin{equation*}
                    r_e = r \cdot (m + \frac{i}{2} \cdot (m + 1)) \Rightarrow R_n = r_e \frac{q^n-1}{q-1} = r \cdot (m + \frac{i}{2} \cdot (m + 1)) \cdot \frac{q^n-1}{q-1}
                \end{equation*}
                Wobei das \(m\) die Anzahl der Ratenzahlungen pro Zinsperiode ist.
        
        \subsection{Tilgungsrechnung}
        Bei der Tilgunsrechnung geht es idR. umd die Rückzahlung eines Krdited (bzw. allgemein Schulden). Die Annuität bezeichnet damit den Betrag, der periodisch zurückgezahlt werden muss.
        Die Annuität setzt sich wiederum aus der Tilgungsrate (Betrag der zurückgezahlt werden soll) und dem Zinsbestandteil (Zinsen die pro Periode von den Restschulden dazukommen).
            \subsubsection{jährliche Ratentilgung}
            Soll die Tilgunsrate jedes Jahr (Periode) gleich bleiben ergibt sich folgende Formel:
            \begin{equation*}
                T = \frac{S}{n}
            \end{equation*}
            Wobei \(T\) die Gesamtschuld ist und \(n\) die Anzahl der Jahre (perioden) der Kreditlaufzeit.\\
            Will man die Zinsrate für ein bestimmtes Jahr (hier \(r\)) ausrechnen gilt folgende Formel: 
            \begin{equation*}
                Z_r = T \cdot (n - r + 1) \cdot i
            \end{equation*} 
            \(i\) ist dabei wieder der Zinsatz, n Die Kreditlaufzeit und \(T\) die Tilgungsrate. \\
            Die Annuität ergibt sich dann wenn man die beiden speziellen Formeln nun in die allgemeine Formel von oben einsetzt und umforfmt: 
            \begin{equation*}
                A = T \cdot (1 + (n - r + 1)i)
            \end{equation*}

            \subsubsection{unterjährliche Ratentilgung}
            Soll mehrmals im Jahr getilgt werden muss der jährliche Zinssatz durch die Anzahl der Tilgungen im Jahr geteilt werden. Weiter beschreibt das \(n\) nicht mehr das ahr sondern die Anzahl
            der Zinsperioden.  
    \section{Kombinatorik}
    \section{Wahrscheinlichkeitstheorie}
    \section{Zufallsvariable und statistische Verteilung}
    \section{Induktive Statistik und statistische Tests}
\end{document}