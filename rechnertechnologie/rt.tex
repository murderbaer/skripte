\documentclass[a4paper]{scrartcl}
\author{Angewandte Informatik}
\title{Grundlagen der Informatik}
%Language specific formating options
\usepackage[utf8]{inputenc} % use utf8 file encoding for TeX sources
\usepackage[T1]{fontenc}    % avoid garbled Unicode text in pdf
\usepackage[english]{babel} % english grammar so words can go in two lines  

\usepackage{amsmath}
\usepackage{circuitikz}
\usepackage{longtable}
\date{WS 20/21}
\begin{document}
    \maketitle
    \newpage
    \tableofcontents
    \newpage
    \section{Intro}
        \subsection{Representation of numbers characters}
        Numbers and characters are saved in the memory and need a binary representation. There are different ways how one can represent numbers and charachters,
        depending on the needs the program has. Having a programm which needs a counter, only nedds positive integers so there is no need for saving decimals.  Also the 
        range is important. Is the programm counting to 100 or 100 milion. Different datatypes need less bytes to store data, but then the range or (Genauigketi) suffers.
        
        \subsubsection*{Integers}
        With unsigned (only positive) integers only differ in how many bits they use. Typicall sizes are  8-bit (short), 16-bit (half word), 32-bit (word) and 64-bit (double word). \\
        Signed integers need to save the minus symbol somewhere. There are several options to "save the minus". One is just saying if the MSB is 1, the number is negative. The problem
        is that \(0000\) and \(1000\) are both \(0\), but one is a positive and one is a negative \(0\) which isn't very effictive. \\
        Another implementation is the one-complement.
        Here you just invert every bit to get the "negative version" of the number. Again the \(\pm 0\) is possible, but the one-complement 
        creates a symmetrie with negative and ppostive numbers and is needed for the two-complement. 
        The two-complement takes the result from the one complement and adds \(+1\) to it. The symmetrie is gone but the \(\pm 0\) is gone (only positve 0) and an extra negative numebr is won. \\
        One other way to create negative nummber is by using a bias/offset. One needs to define the offset first. Now every number in the memory will be read nad the offset will be 
        subtracted from it. An offset of 128 means that the postive numbers will start at \(1000.0000_b \)\footnote{\(1000.0000_b \equiv 0\)}. The offset is used in floating point numbers for the exponent.

        \subsection*{Decimal numbers}
        Decimal numbers also have different possible representation. An easy with a fixed point. The number is treated as an integer but at a specific bit, the point is 
        set. The position of the point needs to be definded first. If there are 8-bit to save the number and the point is defined at bit 3, there will be 5 bits for the integer
        and 3 bits for the mantissa\footnote{Nachkommastellen}. The probelm is, that very big numbers or very small numbers aren't possible. \\
        Floating point numbers fix this by introducing an exponent to the number. The exponetn has an offset, so it can be negativ. A neagtive exponent makes very small numbers
        possible, but because the exponent can be positive aswell big numbers are possible too. The formular fot calculating a normalized flaot is:
        \begin{equation*}
            f = (-1)^{\text{sign}} \cdot 1.\text{mantissa} \cdot 2^{\text{exponent}}
        \end{equation*}  
        Depening on how many bits the float uses, different values need to be inserted into the formular.
        \begin{center}\begin{tabular}{|c|c|c|c|c|}
            \hline
             & sign & mantissa & offset & exponent \\
            \hline
            32-bit & 1-bit & 23-bit & 127-bit & 8-bit \\
            \hline
            64-bit & 1-bit & 52-bit & 1023-bit & 11-bit \\
            \hline
        \end{tabular}\end{center}
        
    \section{Boolean Algebra}
        % Boolean algbra takes (binary) parameters and depending on the operation delivers a \(1\) or a \(0\) as the result.
        % Some of the basic operands are: 
        % \begin{center}
        %     \begin{tabular}{|c|c||L|c|}
        %         & NOT & AND & OR \\
        %         description & inverts the bit it gets & ouptut is one, when all bits are one & result is one, when at least one bit is one\\
        %         notation & \(\overline{1} = 0\) and \(\overline{\overline{A}} = A \)& Y = A * B & A + B \\
        %         gate & \begin{circuitikz}
        %             \draw 
        %             (0,0) node[not port] () {};
        %         \end{circuitikz} 
        %         &  \begin{circuitikz}
        %             \draw 
        %             (0,0) node[and port] () {};
        %         \end{circuitikz}
        %          &
        %          \begin{circuitikz}
        %             \draw 
        %             (0,0) node[or port] () {};
        %         \end{circuitikz}\\
        %     \end{tabular}
        % \end{center}
        \begin{center}
            
            \begin{circuitikz}
                \draw
                (0,0) node[not port] () {}
                (5,0) node[and port] () {}
                (10,0) node[or port] () {};
            \end{circuitikz}
        \end{center}
        
    \section{Instruction Set Archetecture (ISA)}
        The ISA contatins definitions for numbers, adressing and instructions. Numbers were already done in the 1. chapter so they won't be done again.
        \subsection{Adressing}
        Adressing is needed for several operations. Mainly to write results(of an summation), read operands(like summands of a sum) and adresses can be a target of an 
        (un-)conditional branch(if statement, return etc.).  
    \section{miscelanious}
    \subsection{ASCII tabel}
        \begin{longtable}{|c|c|c|c||c|c|c|c|}
        \hline
        Dez & Hex & Okt & Zeichen & Dez & Hex & Okt & Zeichen\\
        \hline
        0 & 0x00 & 000 & NUL & 32 & 0x20 & 040 & SP\\
        1 & 0x01 & 001 & SOH & 33 & 0x21 & 041 & ! \\
        2 & 0x02 & 002 & STX & 34 & 0x22 & 042 & "'\\
        3 & 0x03 & 003 & ETX & 35 & 0x23 & 043 & \# \\
        4 & 0x04 & 004 & EOT & 36 & 0x24 & 044 & \$ \\
        5 & 0x05 & 005 & ENQ & 37 & 0x25 & 045 & \% \\
        6 & 0x06 & 006 & ACK & 38 & 0x26 & 046 & \& \\
        7 & 0x07 & 007 & BEL & 39 & 0x27 & 047 & ' \\
        8 & 0x08 & 010 & BS & 40 & 0x28 & 050 & (  \\
        9 & 0x09 & 011 & TAB & 41 & 0x29 & 051 &  ) \\
        10 & 0x0A & 012 & LF & 42 & 0x2A & 052 & * \\
        11 & 0x0B & 013 & VT & 43 & 0x2B & 053 & + \\
        12 & 0x0C & 014 & FF & 44 & 0x2C & 054 & , \\
        13 & 0x0D & 015 & CR & 45 & 0x2D & 055 & - \\
        14 & 0x0E & 016 & SO & 46 & 0x2E & 056 & . \\
        15 & 0x0F & 017 & SI & 47 & 0x2F & 057 & / \\
        16 & 0x10 & 020 & DLE & 48 & 0x30 & 060 & 0 \\
        17 & 0x11 & 021 & DC1 & 49 & 0x31 & 061 & 1 \\
        18 & 0x12 & 022 & DC2 & 50 & 0x32 & 062 & 2 \\
        19 & 0x13 & 023 & DC3 & 51 & 0x33 & 063 & 3 \\
        20 & 0x14 & 024 & DC4 & 52 & 0x34 & 064 & 4 \\
        21 & 0x15 & 025 & NAK & 53 & 0x35 & 065 & 5 \\
        22 & 0x16 & 026 & SYN & 54 & 0x36 & 066 & 6 \\
        23 & 0x17 & 027 & ETB & 55 & 0x37 & 067 & 7 \\
        24 & 0x18 & 030 & CAN & 56 & 0x38 & 070 & 8 \\
        25 & 0x19 & 031 & EM & 57 & 0x39 & 071 & 9 \\
        26 & 0x1A & 032 & SUB & 58 & 0x3A & 072 & : \\
        27 & 0x1B & 033 & ESC & 59 & 0x3B & 073 & ; \\
        28 & 0x1C & 034 & FS & 60 & 0x3C & 074 & "< \\
        29 & 0x1D & 035 & GS & 61 & 0x3D & 075 & =\\
        30 & 0x1E & 036 & RS & 62 & 0x3E & 076 & "> \\
        31 & 0x1F & 037 & US & 63 & 0x3F & 077 & ? \\
        \hline
        \end{longtable}

        \begin{longtable}{|c|c|c|c||c|c|c|c|}
        \hline
        Dez & Hex & Okt & Zeichen & Dez & Hex & Okt & Zeichen\\
        \hline
        64 & 0x40 & 100 & @ & 96 & 0x60 & 140 & ` \\
        65 & 0x41 & 101 & A & 97 & 0x61 & 141 & a \\
        66 & 0x42 & 102 & B & 98 & 0x62 & 142 & b \\
        67 & 0x43 & 103 & C & 99 & 0x63 & 143 & c \\
        68 & 0x44 & 104 & D & 100 & 0x64 & 144 & d \\
        69 & 0x45 & 105 & E & 101 & 0x65 & 145 & e \\
        70 & 0x46 & 106 & F & 102 & 0x66 & 146 & f \\
        71 & 0x47 & 107 & G & 103 & 0x67 & 147 & g \\
        72 & 0x48 & 110 & H & 104 & 0x68 & 150 & h \\
        73 & 0x49 & 111 & I & 105 & 0x69 & 151 & i \\
        74 & 0x4A & 112 & J & 106 & 0x6A & 152 & j \\
        75 & 0x4B & 113 & K & 107 & 0x6B & 153 & k \\
        76 & 0x4C & 114 & L & 108 & 0x6C & 154 & l \\
        77 & 0x4D & 115 & M & 109 & 0x6D & 155 & m \\
        78 & 0x4E & 116 & N & 110 & 0x6E & 156 & n \\
        79 & 0x4F & 117 & O & 111 & 0x6F & 157 & o \\
        80 & 0x50 & 120 & P & 112 & 0x70 & 160 & p \\
        81 & 0x51 & 121 & Q & 113 & 0x71 & 161 & q \\
        82 & 0x52 & 122 & R & 114 & 0x72 & 162 & r \\
        83 & 0x53 & 123 & S & 115 & 0x73 & 163 & s \\
        84 & 0x54 & 124 & T & 116 & 0x74 & 164 & t \\
        85 & 0x55 & 125 & U & 117 & 0x75 & 165 & u \\
        86 & 0x56 & 126 & V & 118 & 0x76 & 166 & v \\
        87 & 0x57 & 127 & W & 119 & 0x77 & 167 & w \\
        88 & 0x58 & 130 & X & 120 & 0x78 & 170 & x \\
        89 & 0x59 & 131 & Y & 121 & 0x79 & 171 & y \\
        90 & 0x5A & 132 & Z & 122 & 0x7A & 172 & z \\
        91 & 0x5B & 133 & [ & 123 & 0x7B & 173 & \{ \\
        92 & 0x5C & 134 & $\backslash$ & 124 & 0x7C & 174 & $\mid$\\
        93 & 0x5D & 135 & ] & 125 & 0x7D & 175 & \} \\
        94 & 0x5E & 136 & \^{} & 126 & 0x7E & 176 & "~ \\
        95 & 0x5F & 137 & \_ & 127 & 0x7F & 177 & DEL \\
        \hline
        \end{longtable}
\end{document}